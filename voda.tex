\documentclass[report,11pt]{article}

\usepackage[top=2.5cm, bottom=2.5cm, left=2.5cm, right=2.5cm]{geometry}
\usepackage[fleqn]{amsmath}
\usepackage[utf8]{inputenc}
\usepackage[T1]{fontenc} 
\usepackage{lmodern}
\usepackage{graphicx}
\usepackage{titlesec}
\usepackage{hyperref}
\usepackage{xifthen}
\usepackage{tabularx}
\usepackage{xcolor}

\hypersetup{
    bookmarks=true,         	% show bookmarks bar?
    unicode=true,			% non-Latin characters in Acrobat’s bookmark
    colorlinks=true,       	% false: boxed links; true: colored links
    linkcolor=blue,		% color of internal links (change box color with linkbordercolor)
}


\setcounter{secnumdepth}{0}
\titlespacing*{\subsubsection}{0em}{0em}{0em}
\setlength\parindent{0pt}

\title{Voda}
\author{Dan Princ}

\titleformat*{\section}{\Huge\bfseries}
\titleformat*{\subsection}{\LARGE\bfseries}
\titleformat*{\subsubsection}{\large\bfseries}

\newcommand{\rok}{\clearpage\section}
\newcommand{\akce}{\vspace{3em}\subsection}
\newcommand{\reka}{\subsubsection}
\newcommand{\datum[1]}{%
	\addtocontents{toc}{\hspace{1.5em}#1\par}%
	#1%
}

\newcommand{\info[3]}{%
	\vspace{1.5em}\hspace{-.5em}\begin{tabularx}{0.6\textwidth}{X | l X}%
	\parbox[c]{\hsize}{\reka{#1}} & #2 & \textcolor{blue!40!black}{#3} \\%
	\end{tabularx}%
}

\newcommand{\usek}[2][]{%
	\vspace{0.3em}%
	\textcolor{red!40!black}{%
	\ifthenelse{\isempty{#1}}{\\Úsek: #2}{\\Nasedačka: #1\\Vysedačka: #2}%
	}%
	\vspace{0.3em}%
}

\newcommand{\cara}{%
	\begin{center}\line(1,0){350}\end{center}%
}

\begin{document}
\maketitle
\clearpage
\tableofcontents

\rok{2013}
\akce{Piemnot a Tessin}
\datum[18.5. - 25.5.]

Zájezd se Šerpou.

\info[Vorderrhein]{18.5.}{nižší stav}
\usek[Ilanz - 46.780564, 9.231082]{soutok s Hinterrheinem}

\info[Glenner]{19.5.}{vyšší stav}
\usek[pod jezem - 46.724024, 9.20368]{Vorderrhein}

Vyšší stav, proto sme nasedali pod jezem. Kontinuální WW3, dvě těžší místa WW4, nejsou vidět ze silnice - úzký místo pod mostem se solidníma válcema (čouhali z toho nějaký kusy betonu) a o pár set metrů dál peřej s dvěma válcema. Vhodný prohlídnout.

\info[Albula]{20.5.}{nízký stav}
\usek[Surava - 46.664736, 9.612557]{Tiefencastel - 46.661139, 9.582986}

Walderschlucht, nižší voda, cca hodina. Zadarmo.

\info[Moesa]{20.5.}{$\sim80 m^3/s$, vysoký stav}
\usek[Sorte - 46.293057, 9.180048]{San Vittore - 46.237744, 9.116263}

Hodně velkej stav, první 3km WW4+, zbytek WW3+ se zbytkem zájezdu. Na nasedačce šla voda obouma korytama pod mostem a peřej pod mostem byla zalitá, slušný válce. Nejtěžší místo je cca 300m od nasedačky, peřej končí velkym válcem přes celou řeku, ale pouštěl (měli sme 2 krysy), lepší průjezd byl vlevo u břehu. Pod válcem laguna na pochytání. Následuje jez, průjezd prostředkem. Po pár set metrech prudká zatáčka doprava, nejde moc prohlídnout. Za týhle vody nejlepší průjezd vpravo těsně u břehu pod větvema. Následuje několik stupňů s válcema, lepší průjezd vpravo, za třetím stupněm přejezd doleva. Zbytek už byl na oči.

\info[Verzasca]{21.5.}{$\sim35m^3/s$, vyšší stav}
\usek{střední}

Velkej stav, nandaváme 100m pod mostem pod nepěknym válcem. První peřej dáváme vpravo, dál většinu prohlížíme. Pod některejma stupněma jsou slušný válce, vhodnější průjezd je blízko břehu :) Těžší je peřej se dvěma stupni u konce. Lajna je středem, ale chce to rozestupy, když někdo zůstane ve válci :) Všichni zvedaj, Mišu plave. Následuje stupeň s dlouhym rychlym jazykem pod ním, na konci to žene pod šutr vpravo. Nájezd středem, potom hodně doleva. Tady končíme, dochází morál.

\info[Sesia]{22.5.}{střední stav}
\usek[nad Little Canada]{nějaký město dole :)}

Spodní úsek společně s raftama, WW3, velký vlny. Začínáme malym kaňonkem WW4, jede se levym korytem, vlevo to žene na skálu, ale všechno odtejká. Pohodový ježdění na zablbnutí na vlnách.

\info[Mastallone]{22.5.}{střední stav}
\usek[pod kaňonem - 45.841950, 8.252565]{do vesnice, kde začíná nějakej brutální kaňon}

Nasedáme pod kaňonem do kterýho nevidíme a vstupenka neni vůbec zadarmo. Normálně se prej jezdí, ale nechce se nám to prohlížet. Dál je to WW3+, všechno na oči, pohodový poježdění. Končíme ve vesnici, kde to mizí do kaňonu. Úsek je cca na hodinu. Spíme na parkovišti cca 20km proti proudu, ráno dáváme na zablbnutí Landwasser.

\info[Sorba]{23.5.}{nižší stav}
\usek[Rassa - 45.769395, 8.016367]{vpravo za kamennym mostem nad řekou}

Úsek 2.5 km, WW4+, 2-3 hodiny s prohlíženim. Nasedáme po šestý a neznáme to, takže doufáme, že z toho pude vylezt. Většina úseku je blízko u silnice, takže pohoda. Prohlížíme pár míst, zejména kaňonek kousek pod začátkem, ale všechno jedem.  Vysedáme o chvíli dřív, protože se blíží tma a nechceme spadnout do Devils slide :)

\info[Adda]{24.5.}{?}
\usek{netušim}

Kontinuální cca 3km dlouhej úsek WW3, nějakej normální stav...

\info[Ötz + Inn]{25.5.}{190cm Tumpen}
\usek{spodní}

Cestou domu dáváme spodní Ötz do Innu až na vysedačku Innu. Highlight sou 2 rafťáci na nafukovací palmě uprostřed Innu.

\akce{Balkán}
\datum[12.7. - 28.7.]

Výlet na Balkán, příležitostná voda.

\info[Korana]{13.7.}{malý stav}
\usek[Slunj, $N 45^{\circ}7.2354'~E 15^{\circ}35.3268'$]{most, $N 45^{\circ}15.1602'~E 15^{\circ}32.7486'$}

Na začátku WW2+, dál hodně stupňů do 2m, hezká krajina, želvy. Za hodně malý vody 6 hodin. Možnost spaní na vysedačce, louka u řeky.

\info[Zrmanja]{15.7.}{malý stav}
\usek[Raft. centrum, $N 44^{\circ}9.6084'~E 15^{\circ}51.2148'$]{Obrovac, $N 44^{\circ}11.8062'~E 15^{\circ}46.1178'$}

Klasika, drop-pool-drop charakter, do WW3, cca 3 hod. Několik km od začátku 12m vodopád, za malý vody bez šance. Jinak něolik hezkejch stupňů, jeden 4m, ke konci 6m. Hezká krajina, platí se nějaký splutí. 

\info[Neretva]{17.7.}{střední stav}
\usek[Kemp, 43.523582, 18.081626]{most před Konjicem, 43.626343, 18.005236}

Klasika v Bosně, WW3, nádhernej kaňon, několik trochu těžších peřejí, ale všechno jde jet na oči. Za splutí se neplatí, i když se na místě pohybujou ``výběrčí''. Přejeli sme most a ve městě už se blbě vysedá. Vhodná vysedačka je most ještě před vjezdem do města. Je to tak na 3 hodiny.

\info[Tara]{19. - 20.7.}{střední stav}
\usek[Žabljak, 43.128458, 19.309996]{most na hranici s Bosnou, Šćepan Polje, 43.348772, 18.845214}

Černá Hora, úsek 68km, za slušný vody se dá stihnout i za jeden den, nicméně pendl je přes 100km a přes velký kopce. Případně se jezdí 2 dny. Platí se slušná pálka za splutí (45E kajak, 75E Outside). Jinak nádherná krajina, WW2-3, posledních cca 15km je až WW3+.\\
Spali sme asi 20km od konce v kempu na pravý straně v Bosně, dobře tam vaří :)


\rok{2014}

\akce{Rakousko I}
\datum[12.4. - 13.4.]

Jarní Rakousko se Štěpánem, Kubou a Ondrou. Ještě skoro nic neteklo, takže sme dali Kopáč a Lammer, kterej byl oficiálně zavřenej.

\info[Koppentraun]{12. a 13.4.}{TODO}
\usek[nádraží Bad Ausse]{most přes řeku na L547}

Klasika, 8km, až WW4. nižší stav, ale všechno v pohodě jetelný, katarakty v podstatě zadarmo.

\info[Lammer]{12. a 13.4.}{60cm Obergau}
\usek{Lammeröffen}

Provez nás Max Eberl, kterýho sme potkali na Koppentraunu. Nižší stav, dost na pohodu. Ráno sme potkali  pražáky (Kuba Suchý,...) a dali sme 2 jízdy, pak ještě jeden Kopáč.

\akce{Rakousko II}
\datum[7.6. - 9.6.]

Vydařenej zájezd s Kubou (rozbitá loď) a Honzou (vyvrknutej kotník a sádra v Lienzu v nemocnici). Všude hrozně moc vody, takže sme nedali ani Virgen, ani Tauernbach, ani Tunel nebo Wasserfallstrecke.

\info[Lammer]{7.6.}{70cm Obergau}
\usek{Lammeröffen}

Zastávka cestou na Isel, spali sme pod mostem a ráno dali pár jízd, hezká voda, všechno na pohodu.

\info[Lieser]{7.6.}{170cm Spittal}
\usek[pod dálničním mostem pod Gmündem]{vodočet nad Spittalem}

Neznali sme to, takže sme se trochu báli popisu v DKV, kde psali extrémní spád, ale nakonec to bylo úplně na pohodu kontinuální WW3, poslední cca kilometr byl WW3+. Pohodová projíždka, necelý 2 hodiny.

\info[Isel]{8.6.}{100cm Hinterbichl}
\usek[parkoviště Hinterbichl]{most Wallhorn}

6km pěkný kontinuální WW3+ vody, sneslo by to určitě i víc vody. Pěkná je peřejka nad mostem, kde se končí. Asi by se dalo jet na další most o cca 1km níž, potřeba někdy zkusit.

\info[Defereggenbach]{8.6.}{83cm Hopfgarten}
\usek[most nad St. Jakobem 46.911488, 12.300187]{most 1km pod St. Jakobem}

Hodně solidní stav, cca 34 kubíků, jeli sme jenom já a Kuba. Hodně kontinuální 4km sme dali asi za 16 minut, asi tak WW4+. Na první jízdu sme se trochu báli a šolíchali to u břehu, ale všechno se dalo na pohodu, akorát tam člověk nechtěl zaplavat. Ve vesnici něco staví a sou tam lana přes řeku. 

\info[Defereggenbach]{9.6.}{$\sim$80cm Hopfgarten}
\usek{Horní}

Nasedli sme pod vodopádem a chtěli sme vysedat na silničním mostu. Celý sme to prošli a nikde žádný stromy, celý to teklo docela z kopce, asi 3 - 4 km WW4+. Po cca 300m Kuba zaplaval a nechytli sme loď, tu sme dostihli až přesně na vysedačce s dírou na špičce i na zádi. Tim sme skončili tenhle vydařenej zájezd, nicméně úsek za sjetí určitě stojí, tenhle stav byla taková ideální střední voda. Dá se jet i úsek na vodopádem, ale obtížně by se to prohlíželo.


\akce{Isel a okolí}
\datum[27.6. - 6.7.]

Isel s rodinou, kde se měl připojit Kuba a měli sme dát něco solidnějšího, bohužel mu to nevyšlo. Naštěstí sem tam potkal partičku z Čech, který sem vzal aspoň na Standard na Defáči, horní Isel a na úsek pod kataraktem.

\info[Lammer]{28.6.}{50cm Obergau}
\usek{Lammeröffen}

Zastávka cestou na Isel, opravdu malá voda, spodní hranice pro jetí. Taťka si zaplaval na peřeji pod kataraktem a přenes poslední místo, který je za tohodle stavu fakt úzký, takže jenom jedna jízda...

\info[Lieser]{28.6.}{158cm Spittal}
\usek[pod dálničním mostem pod Gmündem]{vodočet nad Spittalem}

Docela malá voda, vyhejbání šutrům, pohodový WW3. Sjel to i Jirka bez problémů.

\info[Möll]{29.6.}{128cm Winklern}
\usek{Klasika od jezu do Winklernu}

Klasika s nafukovačkama, WW2 - 3, 2h.

\info[Gail]{30.6.}{93cm Mauthen}
\usek[Birnbaum]{Mauthen}

Asi největší voda na Gailu co sme měli, pohodový svezení WW2+.


\info[Defereggenbach]{30.6.}{76cm Hopfgarten}
\usek{Standardstrecke}

Standardstrecke za docela hezký vody, až WW4, ale všechno na pohodu. Honza na konci na dojezdu přišel o pádlo a krysil, pádlo už sme nenašli, odplavalo po proudu. Jinak v pohodě. Večer sme dali eště Isel z Hubenu.

\info[Isel]{1.7.}{95cm Hinterbichl}
\usek{horní a spodek od kataraktu}

Na pohodu svezení, ale docela málo vody, blízko spodního limitu.\\
Odpoledne sme to nahodili asi 500m pod kataraktem, hezký svezení WW3+, velký vlny, ale žádný větší válce. Za tohodle stavu (asi 270cm Lienz) by to šlo nahodit i těsně pod katarakt, ale na to sme byli ve slabší sestavě.

\info[Gail]{2.7.}{66cm Maria Luggau}
\usek[Maria Luggau]{Birnbaum}

Poprví sem jel horní Gail, docela dobrej stav, ale mohlo by bejt i víc. Pár set metrů po startu je jedno obtížnější místo s velkym rozdělovákem a válcem před nim, ale jinak je to WW3. Části sou v kaňonu, je třeba dávat pozor na spadlý stromy. Jeden zával přenášíme, jinak je to ale dobrý. Ke konci úseku se objeví 2 těžší peřeje, ale na oči. S občasnym prohlíženim jedeme cca 3.5 hodiny, ale dalo by se to za 2.
\cara
Další den Gail ještě jednou s nafukovačkama, potom jezdíme už jenom spodní Isel a cestou dobu spodní Saalach, kterej fakt už jet nechci.


\end{document}