\documentclass[report,11pt]{article}

\usepackage[top=2.5cm, bottom=2.5cm, left=2.5cm, right=2.5cm]{geometry}
\usepackage[fleqn]{amsmath}
\usepackage[utf8]{inputenc}
\usepackage[T1]{fontenc} 
\usepackage{lmodern}
\usepackage{graphicx}
\usepackage{titlesec}
\usepackage{hyperref}
\usepackage{xifthen}
\usepackage{tabularx}
\usepackage{xcolor}
\usepackage{glossaries}

\hypersetup{
    bookmarks=true,         	% show bookmarks bar?
    unicode=true,			% non-Latin characters in Acrobat’s bookmark
    colorlinks=true,       	% false: boxed links; true: colored links
    linkcolor=blue,		% color of internal links (change box color with linkbordercolor)
}

\setcounter{secnumdepth}{0}
\titlespacing*{\subsubsection}{0em}{0em}{0em}
\setlength\parindent{0pt}

\title{Voda}
\author{Dan Princ}

\titleformat*{\section}{\Huge\bfseries}
\titleformat*{\subsection}{\LARGE\bfseries}
\titleformat*{\subsubsection}{\large\bfseries}

\newcommand{\rok}{\clearpage\section}
\newcommand{\akce}{\vspace{3em}\subsection}
\newcommand{\reka}{\subsubsection}
\newcommand{\datum[1]}{%
	\addtocontents{toc}{\hspace{1.5em}#1\par}%
	#1%
}


\makeglossaries


\newcommand*{\newterm}[2][]{%
\newglossaryentry{#2}%
{name={#2},description={\nopostdesc}}}

\newcommand{\mygls}[1]{\ifglsentryexists{#1}{\gls{#1}}{\textcolor{blue}{\underline{#1}}}}
%\newcommand{\mygls}[1]{\ifglsentryexists{#1}{\gls{#1}}{\newterm{#1}\gls{#1}}}

\newcommand*{\info[3]}{%
	\vspace{1.5em}\hspace{-.5em}\begin{tabularx}{0.6\textwidth}{X | l X}%
	\parbox[c]{\hsize}{\reka{\mygls{#1}}} & #2 & \textcolor{blue!40!black}{#3} \\%
	\end{tabularx}%
}

\newcommand{\usek}[2][]{%
	\vspace{0.3em}%
	\textcolor{red!40!black}{%
	\ifthenelse{\isempty{#1}}{\\Úsek: #2}{\\Nasedačka: #1\\Vysedačka: #2}%
	}%
	\vspace{0.3em}%
}

\newcommand{\cara}{%
	\begin{center}\line(1,0){350}\end{center}%
}


\begin{document}
\maketitle
\clearpage
\tableofcontents

\clearpage
\printglossaries

\rok{2013}
\akce{Piemnot a Tessin}
\datum[18.5. - 25.5.]

Zájezd se Šerpou.

\info[Vorderrhein]{18.5.}{nižší stav}
\usek[Ilanz - 46.780564, 9.231082]{soutok s Hinterrheinem}

\info[Glenner]{19.5.}{vyšší stav}
\usek[pod jezem - 46.724024, 9.20368]{Vorderrhein}

Vyšší stav, proto sme nasedali pod jezem. Kontinuální WW3, dvě těžší místa WW4, nejsou vidět ze silnice - úzký místo pod mostem se solidníma válcema (čouhali z toho nějaký kusy betonu) a o pár set metrů dál peřej s dvěma válcema. Vhodný prohlídnout.

\info[Albula]{20.5.}{nízký stav}
\usek[Surava - 46.664736, 9.612557]{Tiefencastel - 46.661139, 9.582986}

Walderschlucht, nižší voda, cca hodina. Zadarmo.

\info[Moesa]{20.5.}{$\sim80 m^3/s$, vysoký stav}
\usek[Sorte - 46.293057, 9.180048]{San Vittore - 46.237744, 9.116263}

Hodně velkej stav, první 3km WW4+, zbytek WW3+ se zbytkem zájezdu. Na nasedačce šla voda obouma korytama pod mostem a peřej pod mostem byla zalitá, slušný válce. Nejtěžší místo je cca 300m od nasedačky, peřej končí velkym válcem přes celou řeku, ale pouštěl (měli sme 2 krysy), lepší průjezd byl vlevo u břehu. Pod válcem laguna na pochytání. Následuje jez, průjezd prostředkem. Po pár set metrech prudká zatáčka doprava, nejde moc prohlídnout. Za týhle vody nejlepší průjezd vpravo těsně u břehu pod větvema. Následuje několik stupňů s válcema, lepší průjezd vpravo, za třetím stupněm přejezd doleva. Zbytek už byl na oči.

\info[Verzasca]{21.5.}{$\sim35m^3/s$, vyšší stav}
\usek{střední}

Velkej stav, nandaváme 100m pod mostem pod nepěknym válcem. První peřej dáváme vpravo, dál většinu prohlížíme. Pod některejma stupněma jsou slušný válce, vhodnější průjezd je blízko břehu :) Těžší je peřej se dvěma stupni u konce. Lajna je středem, ale chce to rozestupy, když někdo zůstane ve válci :) Všichni zvedaj, Mišu plave. Následuje stupeň s dlouhym rychlym jazykem pod ním, na konci to žene pod šutr vpravo. Nájezd středem, potom hodně doleva. Tady končíme, dochází morál.

\info[Sesia]{22.5.}{střední stav}
\usek[nad Little Canada]{nějaký město dole :)}

Spodní úsek společně s raftama, WW3, velký vlny. Začínáme malym kaňonkem WW4, jede se levym korytem, vlevo to žene na skálu, ale všechno odtejká. Pohodový ježdění na zablbnutí na vlnách.

\info[Mastallone]{22.5.}{střední stav}
\usek[pod kaňonem - 45.841950, 8.252565]{do vesnice, kde začíná nějakej brutální kaňon}

Nasedáme pod kaňonem do kterýho nevidíme a vstupenka neni vůbec zadarmo. Normálně se prej jezdí, ale nechce se nám to prohlížet. Dál je to WW3+, všechno na oči, pohodový poježdění. Končíme ve vesnici, kde to mizí do kaňonu. Úsek je cca na hodinu. Spíme na parkovišti cca 20km proti proudu, ráno dáváme na zablbnutí Landwasser.

\info[Sorba]{23.5.}{nižší stav}
\usek[Rassa - 45.769395, 8.016367]{vpravo za kamennym mostem nad řekou}

Úsek 2.5 km, WW4+, 2-3 hodiny s prohlíženim. Nasedáme po šestý a neznáme to, takže doufáme, že z toho pude vylezt. Většina úseku je blízko u silnice, takže pohoda. Prohlížíme pár míst, zejména kaňonek kousek pod začátkem, ale všechno jedem.  Vysedáme o chvíli dřív, protože se blíží tma a nechceme spadnout do Devils slide :)

\info[Adda]{24.5.}{?}
\usek{netušim}

Kontinuální cca 3km dlouhej úsek WW3, nějakej normální stav...

\info[Otz]{25.5.}{190cm Tumpen} %otz + inn
\usek{spodní}

Cestou domu dáváme spodní Ötz do Innu až na vysedačku Innu. Highlight sou 2 rafťáci na nafukovací palmě uprostřed Innu.

\akce{Balkán}
\datum[12.7. - 28.7.]

Výlet na Balkán, příležitostná voda.

\info[Korana]{13.7.}{malý stav}
\usek[Slunj, $N 45^{\circ}7.2354'~E 15^{\circ}35.3268'$]{most, $N 45^{\circ}15.1602'~E 15^{\circ}32.7486'$}

Na začátku WW2+, dál hodně stupňů do 2m, hezká krajina, želvy. Za hodně malý vody 6 hodin. Možnost spaní na vysedačce, louka u řeky.

\info[Zrmanja]{15.7.}{malý stav}
\usek[Raft. centrum, $N 44^{\circ}9.6084'~E 15^{\circ}51.2148'$]{Obrovac, $N 44^{\circ}11.8062'~E 15^{\circ}46.1178'$}

Klasika, drop-pool-drop charakter, do WW3, cca 3 hod. Několik km od začátku 12m vodopád, za malý vody bez šance. Jinak něolik hezkejch stupňů, jeden 4m, ke konci 6m. Hezká krajina, platí se nějaký splutí. 

\info[Neretva]{17.7.}{střední stav}
\usek[Kemp, 43.523582, 18.081626]{most před Konjicem, 43.626343, 18.005236}

Klasika v Bosně, WW3, nádhernej kaňon, několik trochu těžších peřejí, ale všechno jde jet na oči. Za splutí se neplatí, i když se na místě pohybujou ``výběrčí''. Přejeli sme most a ve městě už se blbě vysedá. Vhodná vysedačka je most ještě před vjezdem do města. Je to tak na 3 hodiny.

\info[Tara]{19. - 20.7.}{střední stav}
\usek[Žabljak, 43.128458, 19.309996]{most na hranici s Bosnou, Šćepan Polje, 43.348772, 18.845214}

Černá Hora, úsek 68km, za slušný vody se dá stihnout i za jeden den, nicméně pendl je přes 100km a přes velký kopce. Byli sme jenom jedno auto, takže sme zaplatili Defendera, co nás 4 vzal i s loděma na nasedačku za 100E. Řeku sme nakonec dali jako dvou denní výlet. Platí se slušná pálka za splutí (45E kajak, 75E Outside). Jinak nádherná krajina, WW2-3, posledních cca 15km je až WW3+.\\
Spali sme asi 20km od konce v kempu na pravý straně v Bosně, dobře tam vaří :)

\cara
Cestou zpátky sme dali spodní Isel za stavu kolem 300cm a Möll asi za 130cm.


\rok{2014}

\akce{Rakousko I}
\datum[12.4. - 13.4.]

Jarní Rakousko se Štěpánem, Kubou a Ondrou. Ještě skoro nic neteklo, takže sme dali Kopáč a Lammer, kterej byl oficiálně zavřenej.

\info[Koppentraun]{12. a 13.4.}{105cm Obertraun}
\usek[nádraží Bad Ausse]{most přes řeku na L547}

Klasika, 8km, až WW4. nižší stav, ale všechno v pohodě jetelný, katarakty v podstatě zadarmo.

\info[Lammer]{12. a 13.4.}{60cm Obergau}
\usek{Lammeröffen}

Provez nás Max Eberl, kterýho sme potkali na Koppentraunu. Nižší stav, dost na pohodu. Ráno sme potkali  pražáky (Kuba Suchý,...) a dali sme 2 jízdy, pak ještě jeden Kopáč.

\akce{Rakousko II}
\datum[7.6. - 9.6.]

Vydařenej zájezd s Kubou (rozbitá loď) a Honzou (vyvrknutej kotník a sádra v Lienzu v nemocnici). Všude hrozně moc vody, takže sme nedali ani Virgen, ani Tauernbach, ani Tunel nebo Wasserfallstrecke. Byli sme taky prohlídnout Untertalbach, měli sme dobrej stav, ale klukům se do toho nechtělo... já bych dal. Takže v podstatě sme nic nejeli, ale rozbili loď a navštívili nemocnici...

\info[Lammer]{7.6.}{70cm Obergau}
\usek{Lammeröffen}

Zastávka cestou na Isel, spali sme pod mostem a ráno dali pár jízd, hezká voda, všechno na pohodu.

\info[Lieser]{7.6.}{170cm Spittal}
\usek[pod dálničním mostem pod Gmündem]{vodočet nad Spittalem}

Neznali sme to, takže sme se trochu báli popisu v DKV, kde psali extrémní spád, ale nakonec to bylo úplně na pohodu kontinuální WW3, poslední cca kilometr byl WW3+. Pohodová projíždka, necelý 2 hodiny.

\info[Isel]{8.6.}{100cm Hinterbichl}
\usek[parkoviště Hinterbichl]{most Wallhorn}

6km pěkný kontinuální WW3+ vody, sneslo by to určitě i víc vody. Pěkná je peřejka nad mostem, kde se končí. Asi by se dalo jet na další most o cca 1km níž, potřeba někdy zkusit.

\info[Defereggenbach]{8.6.}{83cm Hopfgarten}
\usek[most nad St. Jakobem 46.911488, 12.300187]{most 1km pod St. Jakobem}

Hodně solidní stav, cca 34 kubíků, jeli sme jenom já a Kuba. Hodně kontinuální 4km sme dali asi za 16 minut, asi tak WW4+. Na první jízdu sme se trochu báli a šolíchali to u břehu, ale všechno se dalo na pohodu, akorát tam člověk nechtěl zaplavat. Ve vesnici něco staví a sou tam lana přes řeku. 

\info[Defereggenbach]{9.6.}{$\sim$80cm Hopfgarten}
\usek{Horní}

Nasedli sme pod vodopádem a chtěli sme vysedat na silničním mostu. Celý sme to prošli a nikde žádný stromy, celý to teklo docela z kopce, asi 3 - 4 km WW4+. Po cca 300m Kuba zaplaval a nechytli sme loď, tu sme dostihli až přesně na vysedačce s dírou na špičce i na zádi. Tim sme skončili tenhle vydařenej zájezd, nicméně úsek za sjetí určitě stojí, tenhle stav byla taková ideální střední voda. Dá se jet i úsek na vodopádem, ale obtížně by se to prohlíželo.


\akce{Isel a okolí}
\datum[27.6. - 6.7.]

Isel s rodinou, kde se měl připojit Kuba a měli sme dát něco solidnějšího, bohužel mu to nevyšlo. Naštěstí sem tam potkal partičku z Čech, který sem vzal aspoň na Standard na Defáči, horní Isel a na úsek pod kataraktem.

\info[Lammer]{28.6.}{50cm Obergau}
\usek{Lammeröffen}

Zastávka cestou na Isel, opravdu malá voda, spodní hranice pro jetí. Taťka si zaplaval na peřeji pod kataraktem a přenes poslední místo, který je za tohodle stavu fakt úzký, takže jenom jedna jízda...

\info[Lieser]{28.6.}{158cm Spittal}
\usek[pod dálničním mostem pod Gmündem]{vodočet nad Spittalem}

Docela malá voda, vyhejbání šutrům, pohodový WW3. Sjel to i Jirka bez problémů.

\info[Moll]{29.6.}{128cm Winklern}
\usek{Klasika od jezu do Winklernu}

Klasika s nafukovačkama, WW2 - 3, 2h.

\info[Gail]{30.6.}{93cm Mauthen}
\usek[Birnbaum]{Mauthen}

Asi největší voda na Gailu co sme měli, pohodový svezení WW2+.


\info[Defereggenbach]{30.6.}{76cm Hopfgarten}
\usek{Standardstrecke}

Standardstrecke za docela hezký vody, až WW4, ale všechno na pohodu. Honza na konci na dojezdu přišel o pádlo a krysil, pádlo už sme nenašli, odplavalo po proudu. Jinak v pohodě. Večer sme dali eště Isel z Hubenu.

\info[Isel]{1.7.}{95cm Hinterbichl}
\usek{horní a spodek od kataraktu}

Na pohodu svezení, ale docela málo vody, blízko spodního limitu.\\
Odpoledne sme to nahodili asi 500m pod kataraktem, hezký svezení WW3+, velký vlny, ale žádný větší válce. Za tohodle stavu (asi 270cm Lienz) by to šlo nahodit i těsně pod katarakt, ale na to sme byli ve slabší sestavě.

\info[Gail]{2.7.}{66cm Maria Luggau}
\usek[Maria Luggau]{Birnbaum}

Poprví sem jel horní Gail, docela dobrej stav, ale mohlo by bejt i víc. Pár set metrů po startu je jedno obtížnější místo s velkym rozdělovákem a válcem před nim, ale jinak je to WW3. Části sou v kaňonu, je třeba dávat pozor na spadlý stromy. Jeden zával přenášíme, jinak je to ale dobrý. Ke konci úseku se objeví 2 těžší peřeje, ale na oči. S občasnym prohlíženim jedeme cca 3.5 hodiny, ale dalo by se to za 2.
\cara
Další den Gail ještě jednou s nafukovačkama, potom jezdíme už jenom spodní Isel a cestou dobu spodní Saalach, kterej fakt už jet nechci.


\akce{Rakousko III}
\datum[22.8. - 26.8.]

Rakousko s Kubou, Patem a Šuhym, směr Inn, Ötz, Isel.


\info[Inn]{22.8.}{25 $m^{3}_{s}$}
\usek{Giarsun + Ardez, nasedačka vesnice Giarsun, vysedačka $N 46^{\circ}47.0568'~E 10^{\circ}15.1692'$}

Giarsun je na pohodu, za tohodle stavu WW3-4, asi dvě místa sme prohlídli, jinak na oči. Kuba to jel prej i za 80 kubíků, což bude jiná řeka :) Dojede se k mostu, tam začíná Ardez, kterej je o trochu těžší. Stav co sme měli vypadal docela optimálně, takovej střed. V půlce se přenáší na levym břehu cca 100m nejetelná zasifonovaná peřej. Pod tim je těžší místo, který se jede těsně zleva mezi šutrama, potom pořád vlevo.Na konci sou dva stupně, první nižší se jede vlevo, druhej (cca 1.5m) se jede středem. Pak už se to zklidňuje a je to na oči. Když se u vody vlevo objeví silnice, blíží se konec. S občasnym prohlíženim cca 2 hodiny.


\info[Inn]{22.8.}{260cm Kajetan Brücke}
\usek{Tösener schlucht, nasedačka Tösens 47.007214, 10.598180, vysedačka Ried vlevo před mostem}

Pěknej střední stav, 3 těžší peřeje, trochu těžší a svižnější než Imster schlucht, ale všechno na oči. Celkově je to všechno o trochu větší, má to sílu, ale na pohodu. 30 min.

\info[Pitzbach]{23.8}{45cm Wenns}
\usek[most Wiese 47.115556, 10.795853]{47.129180, 10.768366}

U vysedačky je plácek u řeky, kde sme spali. Nasedali sme před nejtěžším úsek u mostu ve Wiese, ale dá se jezdit i o 2km vejš. Stav 45 cm už je spíš nižší, hodně se berou šutry a nedá se jim moc vyhnout, nicméně na rozumný sjetí to stačí. 3-5 cm navíc by byly asi optimální. Vhodný předem prohlídnout kvůli stromům, protože v některejch místech by se zastavovalo hodně těžko.

Nejtěžší místo je hned za nasedačkou, 300m peřej v pravý zatáče, najíždí se vlevo, potom je nutný dostat se napravo na 1m drop, odkud se pokračuje víceméně středem s vodou přes několik válců. Dalších 300m je klidnějších, potom je těžší místo v levý zatáčce, nejdřív dva menší stupně, potom cca 2m stupeň, za kterym je šupna u levýho břehu, docela to frčí.Potom následuje rozlitější jednodušší úsek, až před druhou vesnici, kde je místo se dvěma skokama (viditelný ze silnice), nejlepší jet vlevo. Hned za tim se buduje navigace, takže pozor na bordel ze stavby. Ve vesnici je 2m kolmej jez, vhodný prohlídnout předem ze silnice. Potom opět rozlitější úsek, až před vysedačkou to nabírá opět spád, končí to cca 100m dlouhou těžší peřejí s několika válcema, kterou je vhodný prohlídnout. Potom už je vysedačka, případn je možný jet o 100m dál, nicméně je nutný vylezt před mostem, za nim následuje soutěska.
  
\info[Otz]{23.8}{208cm Tumpen}
\usek{Horní}

Pěkná pohodová voda, je to všechno zalitný, neni se nutný vyhejbat šutrům, ale neobjevujou se ani žádný velký válce. Všechno na pohodu na oči. 25min.

\info[Venter Ache]{24.8}{206cm Tumpen}
\usek{Spodní}

Nasedačka je most, u odbočky z hlavní je směrovka na parkoviště P4. Supr střední stav, všechno na pohodu, nic se nemusí drhnout přes šutry, ale nebyl to ani velkej letní stav s brutálníma válcema... První těžší místo je rozdělovák, jede se vpravo, kde to žene na skálu a dělá to takovej polštář, ideální lajna je co nejblíž šutru vlevo. Nejtěžší místo je 100m peřej mezi druhym a třetim mostem (odbočka u autobusový zastávky Bodenegg), nicméně jede se celý s vodou, v nájezdu dva válce. Pod mostem je blbý místo vlevo u skály. Těžší místo je pravá zatáčka, kde se najíždí proti skále  a dělá se tam boční válec. Nájezd kdekoliv vpravo od středu. Ještě je tam několik těžších míst, ale všude se dá v klidu zastavit a prohlídnout. První jízda s prohlíženim 1h, druhá jízda 30min.

\cara
Následuje přejezd na Isel, jedeme ze Söldenu proti proudu, což je cesta na pass 2500m (platí se 14eur), kde se přejíždí do Itálie. Nicméně v Itálii se sjede do 720m a pak se jede přes další pass 2400m, ze kterýho se sjede na cestu z Brenneru. Kus cesty po brennerstrasse, potom doleva přímo na Defereggenbach z druhý strany, což je opět přes pass cca 2300m. Ten je navíc jednosměrnej, v tomhle směru pouštěj auta mezi 30. a 45. minutou každou hodinu (semafor je u jezera). Tahle cesta opravdu neni optimální, lépe se vrátit a jet přes Brenner, nebo to objet celý přes Rakousko.

\info[Defereggenbach]{25.8.}{62cm Hopfgarten}
\usek{Standardstrecke}

I za tohodle stavu pořád eště v pohodě jetelný a vůbec né nudný. Bohužel na tunel nebo wasserfallstrecke je to pořád moc.

\info[Tauernbach]{25.8.}{160cm Prossegg}
\usek{Standardstrecke}

Střední voda, ale sneslo by to na pohodu i o 5 - 10cm víc. Nasedačka cca 200m nad dřevěnym mostem u vysokýho bočního vodopádu. První kilák v pohodě, potom cca 800m těžší úsek WW4-5. První místo je taková úzká škvíra s bočnim válcem, určitě prohlídnout. Jedu jako jedinej, nájezd z pravýho středu do leva, ve výsledku  průjezd úplně v pohodě. Hned potom následuje hodně kamenitá peřej v levý zatáčce, jede se středem s vodou. Potom několik peřejí a stupňů, dá se jet na oči. Poslední prudší stupeň vhodný jet spíš vpravo. Poslední těžší místo je 1m stupeň pod mostem, vhodný prohlídnout z levýho břehu nebo předem z cesty. Vysedačka kdekoliv za mostem, po cca 400m vožný lesem vylezt na parkoviště u odbočky z hlavní silnice. Nutný zastavit tady, následuje těžkej kaňon zapadanej stromama.

\info[Isel]{25.8.}{280cm Lienz}
\usek{od kataraktu do Hubenu}

Spíme u řeky pod kataraktem na levym břehu, asfaltová odbočka u vesnice Feld (46.956511, 12.560980). Nasedáme těsně pod kataraktem. První peřeje trochu těžší, WW4, potom se to trochu zklidňuje. Za tohodle stavu na pohodu poježdění, na vyblbnutí, žádný nebezpečný místo. Hořejšek je čerstvě převoranej, zmizely velký kusy břehů a řeka místy teče úplně jinudy než dřív. Až na plážičku pod Hubenem dáváme za 23min. Jedeme 3 jízdy během dvou dní...




\end{document}