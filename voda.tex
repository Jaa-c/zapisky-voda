\documentclass[report,11pt]{article}

\usepackage[top=2.5cm, bottom=2.5cm, left=2.5cm, right=2.5cm]{geometry}
\usepackage[fleqn]{amsmath}
\usepackage[utf8]{inputenc}
\usepackage[T1]{fontenc} 
\usepackage{lmodern}
\usepackage{graphicx}
\usepackage{titlesec}
\usepackage{hyperref}
\usepackage{xifthen}
\usepackage{tabularx}
\usepackage{xcolor}
\usepackage{glossaries}

\hypersetup{
    bookmarks=true,         	% show bookmarks bar?
    unicode=true,			% non-Latin characters in Acrobat’s bookmark
    colorlinks=true,       	% false: boxed links; true: colored links
    linkcolor=blue,		% color of internal links (change box color with linkbordercolor)
}

\setcounter{secnumdepth}{0}
\titlespacing*{\subsubsection}{0em}{0em}{0em}
\setlength\parindent{0pt}

\title{Voda}
\author{Dan Princ}

\titleformat*{\section}{\Huge\bfseries}
\titleformat*{\subsection}{\LARGE\bfseries}
\titleformat*{\subsubsection}{\large\bfseries}

\newcommand{\rok}{\clearpage\section}
\newcommand{\akce}{\vspace{3em}\subsection}
\newcommand{\reka}{\subsubsection}
\newcommand{\datum[1]}{%
	\addtocontents{toc}{\hspace{1.5em}#1\par}%
	#1%
}


\makeatletter
\newcommand{\glssubsection}[1]{%
  \@ifundefined{@glssubsection\detokenize{#1}}
    {\protected@write\@auxout{}{\string\newglossaryentry{#1}{name=#1,description={}}}%
     \global\@namedef{@glssubsection\detokenize{#1}}{}}{}%
  \glsdoifexists{#1}
    {\subsection{\texorpdfstring{\Gls{#1}}{#1}}}%
    {\subsection{#1}}%
}
\makeatother

\makeglossaries


\newcommand*{\newterm}[2][]{%
\newglossaryentry{#2}%
{name={#2},description={\nopostdesc}}}

%prvni je pro vygereovani glossaries, druhý je pro latetex po zavolání -makeglossaries voda- v bashi
\newcommand{\mygls}[1]{\ifglsentryexists{#1}{\gls{#1}}{\textcolor{blue}{\underline{#1}}}}
%\newcommand{\mygls}[1]{\ifglsentryexists{#1}{\gls{#1}}{\newterm{#1}\gls{#1}}}

\newcommand*{\info[3]}{%
	\vspace{1.5em}\hspace{-.5em}\begin{tabularx}{0.9\textwidth}{X | l X}%
	\parbox[c]{\hsize}{\reka{\mygls{#1}}} & #2 & \textcolor{blue!40!black}{#3} \\%
	\end{tabularx}%
}

\newcommand{\usek}[2][]{%
	\vspace{0.3em}%
	\textcolor{red!40!black}{%
	\ifthenelse{\isempty{#1}}{\\Úsek: #2}{\\Nasedačka: #1\\Vysedačka: #2}%
	}%
	\vspace{0.3em}%
}

\newcommand{\cara}{%
	\begin{center}\line(1,0){350}\end{center}%
}


\begin{document}
\maketitle
\clearpage
\tableofcontents

\clearpage
\printglossaries

\rok{2013}
\akce{Piemnot a Tessin}
\datum[18.5. - 25.5.]

Zájezd se Šerpou.

\info[Vorderrhein]{18.5.}{nižší stav}
\usek[Ilanz - 46.780564, 9.231082]{soutok s Hinterrheinem}

\info[Glenner]{19.5.}{vyšší stav}
\usek[pod jezem - 46.724024, 9.20368]{Vorderrhein}

Vyšší stav, proto sme nasedali pod jezem. Kontinuální WW3, dvě těžší místa WW4, nejsou vidět ze silnice - úzký místo pod mostem se solidníma válcema (čouhali z toho nějaký kusy betonu) a o pár set metrů dál peřej s dvěma válcema. Vhodný prohlídnout.

\info[Albula]{20.5.}{nízký stav}
\usek[Surava - 46.664736, 9.612557]{Tiefencastel - 46.661139, 9.582986}

Walderschlucht, nižší voda, cca hodina. Zadarmo.

h\usek[Sorte - 46.293057, 9.180048]{San Vittore - 46.237744, 9.116263}

Hodně velkej stav, první 3km WW4+, zbytek WW3+ se zbytkem zájezdu. Na nasedačce šla voda obouma korytama pod mostem a peřej pod mostem byla zalitá, slušný válce. Nejtěžší místo je cca 300m od nasedačky, peřej končí velkym válcem přes celou řeku, ale pouštěl (měli sme 2 krysy), lepší průjezd byl vlevo u břehu. Pod válcem laguna na pochytání. Následuje jez, průjezd prostředkem. Po pár set metrech prudká zatáčka doprava, nejde moc prohlídnout. Za týhle vody nejlepší průjezd vpravo těsně u břehu pod větvema. Následuje několik stupňů s válcema, lepší průjezd vpravo, za třetím stupněm přejezd doleva. Zbytek už byl na oči.

\info[Verzasca]{21.5.}{$\sim35m^3/s$, vyšší stav}
\usek{střední}

Velkej stav, nandaváme 100m pod mostem pod nepěknym válcem. První peřej dáváme vpravo, dál většinu prohlížíme. Pod některejma stupněma jsou slušný válce, vhodnější průjezd je blízko břehu :) Těžší je peřej se dvěma stupni u konce. Lajna je středem, ale chce to rozestupy, když někdo zůstane ve válci :) Všichni zvedaj, Mišu plave. Následuje stupeň s dlouhym rychlym jazykem pod ním, na konci to žene pod šutr vpravo. Nájezd středem, potom hodně doleva. Tady končíme, dochází morál.

\info[Sesia]{22.5.}{střední stav}
\usek[nad Little Canada - asi 45.811830, 8.129362]{nějaký město dole :)}

Spodní úsek společně s raftama, WW3, velký vlny. Začínáme malym kaňonkem WW4, jede se levym korytem, vlevo to žene na skálu, ale všechno odtejká. Pohodový ježdění na zablbnutí na vlnách.

\info[Mastallone]{22.5.}{střední stav}
\usek[pod kaňonem - 45.841950, 8.252565]{do vesnice, kde začíná nějakej brutální kaňon}

Nasedáme pod kaňonem do kterýho nevidíme a vstupenka neni vůbec zadarmo. Normálně se prej jezdí, ale nechce se nám to prohlížet. Dál je to WW3+, všechno na oči, pohodový poježdění. Končíme ve vesnici, kde to mizí do kaňonu. Úsek je cca na hodinu. Spíme na parkovišti cca 20km proti proudu, ráno dáváme na zablbnutí Landwasser.

\info[Sorba]{23.5.}{nižší stav}
\usek[Rassa - 45.769395, 8.016367]{vpravo za kamennym mostem nad řekou}

Úsek 2.5 km, WW4+, 2-3 hodiny s prohlíženim. Nasedáme po šestý a neznáme to, takže doufáme, že z toho pude vylezt. Většina úseku je blízko u silnice, takže pohoda. Prohlížíme pár míst, zejména kaňonek kousek pod začátkem, ale všechno jedem.  Vysedáme o chvíli dřív, protože se blíží tma a nechceme spadnout do Devils slide :)

\info[Adda]{24.5.}{?}
\usek{netušim}

Kontinuální cca 3km dlouhej úsek WW3, nějakej normální stav...

\info[Otz]{25.5.}{190cm Tumpen} %otz + inn
\usek{spodní}

Cestou domu dáváme spodní Ötz do Innu až na vysedačku Innu. Highlight sou 2 rafťáci na nafukovací palmě uprostřed Innu.

\akce{Balkán}
\datum[12.7. - 28.7.]

Výlet na Balkán, příležitostná voda.

\info[Korana]{13.7.}{malý stav}
\usek[Slunj, $N 45^{\circ}7.2354'~E 15^{\circ}35.3268'$]{most, $N 45^{\circ}15.1602'~E 15^{\circ}32.7486'$}

Na začátku WW2+, dál hodně stupňů do 2m, hezká krajina, želvy. Za hodně malý vody 6 hodin. Možnost spaní na vysedačce, louka u řeky.

\info[Zrmanja]{15.7.}{malý stav}
\usek[Raft. centrum, $N 44^{\circ}9.6084'~E 15^{\circ}51.2148'$]{Obrovac, $N 44^{\circ}11.8062'~E 15^{\circ}46.1178'$}

Klasika, drop-pool-drop charakter, do WW3, cca 3 hod. Několik km od začátku 12m vodopád, za malý vody bez šance. Jinak něolik hezkejch stupňů, jeden 4m, ke konci 6m. Hezká krajina, platí se nějaký splutí. 

\info[Neretva]{17.7.}{střední stav}
\usek[Kemp, 43.523582, 18.081626]{most před Konjicem, 43.626343, 18.005236}

Klasika v Bosně, WW3, nádhernej kaňon, několik trochu těžších peřejí, ale všechno jde jet na oči. Za splutí se neplatí, i když se na místě pohybujou ``výběrčí''. Přejeli sme most a ve městě už se blbě vysedá. Vhodná vysedačka je most ještě před vjezdem do města. Je to tak na 3 hodiny.

\info[Tara]{19. - 20.7.}{střední stav}
\usek[Žabljak, 43.128458, 19.309996]{most na hranici s Bosnou, Šćepan Polje, 43.348772, 18.845214}

Černá Hora, úsek 68km, za slušný vody se dá stihnout i za jeden den, nicméně pendl je přes 100km a přes velký kopce. Byli sme jenom jedno auto, takže sme zaplatili Defendera, co nás 4 vzal i s loděma na nasedačku za 100E. Řeku sme nakonec dali jako dvou denní výlet. Platí se slušná pálka za splutí (45E kajak, 75E Outside). Jinak nádherná krajina, WW2-3, posledních cca 15km je až WW3+.\\
Spali sme asi 20km od konce v kempu na pravý straně v Bosně, dobře tam vaří :)

\cara
Cestou zpátky sme dali spodní Isel za stavu kolem 300cm a Möll asi za 130cm.


\rok{2014}

\akce{Rakousko I}
\datum[12.4. - 13.4.]

Jarní Rakousko se Štěpánem, Kubou a Ondrou. Ještě skoro nic neteklo, takže sme dali Kopáč a Lammer, kterej byl oficiálně zavřenej.

\info[Koppentraun]{12. a 13.4.}{105cm Obertraun}
\usek[nádraží Bad Ausse]{most přes řeku na L547}

Klasika, 8km, až WW4. nižší stav, ale všechno v pohodě jetelný, katarakty v podstatě zadarmo.

\info[Lammer]{12. a 13.4.}{60cm Obergau}
\usek{Lammeröffen}

Provez nás Max Eberl, kterýho sme potkali na Koppentraunu. Nižší stav, dost na pohodu. Ráno sme potkali  pražáky (Kuba Suchý,...) a dali sme 2 jízdy, pak ještě jeden Kopáč.

\akce{Rakousko II}
\datum[7.6. - 9.6.]

Vydařenej zájezd s Kubou (rozbitá loď) a Honzou (vyvrknutej kotník a sádra v Lienzu v nemocnici). Všude hrozně moc vody, takže sme nedali ani Virgen, ani Tauernbach, ani Tunel nebo Wasserfallstrecke. Byli sme taky prohlídnout Untertalbach, měli sme dobrej stav, ale klukům se do toho nechtělo... já bych dal. Takže v podstatě sme nic nejeli, ale rozbili loď a navštívili nemocnici...

\info[Lammer]{7.6.}{70cm Obergau}
\usek{Lammeröffen}

Zastávka cestou na Isel, spali sme pod mostem a ráno dali pár jízd, hezká voda, všechno na pohodu.

\info[Lieser]{7.6.}{170cm Spittal}
\usek[pod dálničním mostem pod Gmündem]{vodočet nad Spittalem}

Neznali sme to, takže sme se trochu báli popisu v DKV, kde psali extrémní spád, ale nakonec to bylo úplně na pohodu kontinuální WW3, poslední cca kilometr byl WW3+. Pohodová projíždka, necelý 2 hodiny.

\info[Isel]{8.6.}{100cm Hinterbichl}
\usek[parkoviště Hinterbichl]{most Wallhorn}

6km pěkný kontinuální WW3+ vody, sneslo by to určitě i víc vody. Pěkná je peřejka nad mostem, kde se končí. Asi by se dalo jet na další most o cca 1km níž, potřeba někdy zkusit.

\info[Defereggenbach]{8.6.}{83cm Hopfgarten}
\usek[most nad St. Jakobem 46.911488, 12.300187]{most 1km pod St. Jakobem}

Hodně solidní stav, cca 34 kubíků, jeli sme jenom já a Kuba. Hodně kontinuální 4km sme dali asi za 16 minut, asi tak WW4+. Na první jízdu sme se trochu báli a šolíchali to u břehu, ale všechno se dalo na pohodu, akorát tam člověk nechtěl zaplavat. Ve vesnici něco staví a sou tam lana přes řeku. 

\info[Defereggenbach]{9.6.}{$\sim$80cm Hopfgarten}
\usek{Horní}

Nasedli sme pod vodopádem a chtěli sme vysedat na silničním mostu. Celý sme to prošli a nikde žádný stromy, celý to teklo docela z kopce, asi 3 - 4 km WW4+. Po cca 300m Kuba zaplaval a nechytli sme loď, tu sme dostihli až přesně na vysedačce s dírou na špičce i na zádi. Tim sme skončili tenhle vydařenej zájezd, nicméně úsek za sjetí určitě stojí, tenhle stav byla taková ideální střední voda. Dá se jet i úsek na vodopádem, ale obtížně by se to prohlíželo.


\akce{Isel a okolí}
\datum[27.6. - 6.7.]

Isel s rodinou, kde se měl připojit Kuba a měli sme dát něco solidnějšího, bohužel mu to nevyšlo. Naštěstí sem tam potkal partičku z Čech, který sem vzal aspoň na Standard na Defáči, horní Isel a na úsek pod kataraktem.

\info[Lammer]{28.6.}{50cm Obergau}
\usek{Lammeröffen}

Zastávka cestou na Isel, opravdu malá voda, spodní hranice pro jetí. Taťka si zaplaval na peřeji pod kataraktem a přenes poslední místo, který je za tohodle stavu fakt úzký, takže jenom jedna jízda...

\info[Lieser]{28.6.}{158cm Spittal}
\usek[pod dálničním mostem pod Gmündem]{vodočet nad Spittalem}

Docela malá voda, vyhejbání šutrům, pohodový WW3. Sjel to i Jirka bez problémů.

\info[Moll]{29.6.}{128cm Winklern}
\usek{Klasika od jezu do Winklernu}

Klasika s nafukovačkama, WW2 - 3, 2h.

\info[Gail]{30.6.}{93cm Mauthen}
\usek[Birnbaum]{Mauthen}

Asi největší voda na Gailu co sme měli, pohodový svezení WW2+.


\info[Defereggenbach]{30.6.}{76cm Hopfgarten}
\usek{Standardstrecke}

Standardstrecke za docela hezký vody, až WW4, ale všechno na pohodu. Honza na konci na dojezdu přišel o pádlo a krysil, pádlo už sme nenašli, odplavalo po proudu. Jinak v pohodě. Večer sme dali eště Isel z Hubenu.

\info[Isel]{1.7.}{95cm Hinterbichl}
\usek{horní a spodek od kataraktu}

Na pohodu svezení, ale docela málo vody, blízko spodního limitu.\\
Odpoledne sme to nahodili asi 500m pod kataraktem, hezký svezení WW3+, velký vlny, ale žádný větší válce. Za tohodle stavu (asi 270cm Lienz) by to šlo nahodit i těsně pod katarakt, ale na to sme byli ve slabší sestavě.

\info[Gail]{2.7.}{66cm Maria Luggau}
\usek[Maria Luggau]{Birnbaum}

Poprví sem jel horní Gail, docela dobrej stav, ale mohlo by bejt i víc. Pár set metrů po startu je jedno obtížnější místo s velkym rozdělovákem a válcem před nim, ale jinak je to WW3. Části sou v kaňonu, je třeba dávat pozor na spadlý stromy. Jeden zával přenášíme, jinak je to ale dobrý. Ke konci úseku se objeví 2 těžší peřeje, ale na oči. S občasnym prohlíženim jedeme cca 3.5 hodiny, ale dalo by se to za 2.
\cara
Další den Gail ještě jednou s nafukovačkama, potom jezdíme už jenom spodní Isel a cestou dobu spodní Saalach, kterej fakt už jet nechci.


\akce{Rakousko III}
\datum[22.8. - 26.8.]

Rakousko s Kubou, Patem a Šuhym, směr Inn, Ötz, Isel.


\info[Inn]{22.8.}{25 $m^{3}_{s}$}
\usek{Giarsun + Ardez, nasedačka vesnice Giarsun, vysedačka $N 46^{\circ}47.0568'~E 10^{\circ}15.1692'$}

Giarsun je na pohodu, za tohodle stavu WW3-4, asi dvě místa sme prohlídli, jinak na oči. Kuba to jel prej i za 80 kubíků, což bude jiná řeka :) Dojede se k mostu, tam začíná Ardez, kterej je o trochu těžší. Stav co sme měli vypadal docela optimálně, takovej střed. V půlce se přenáší na levym břehu cca 100m nejetelná zasifonovaná peřej. Pod tim je těžší místo, který se jede těsně zleva mezi šutrama, potom pořád vlevo.Na konci sou dva stupně, první nižší se jede vlevo, druhej (cca 1.5m) se jede středem. Pak už se to zklidňuje a je to na oči. Když se u vody vlevo objeví silnice, blíží se konec. S občasnym prohlíženim cca 2 hodiny.


\info[Inn]{22.8.}{260cm Kajetan Brücke}
\usek{Tösener schlucht, nasedačka Tösens 47.007214, 10.598180, vysedačka Ried vlevo před mostem}

Pěknej střední stav, 3 těžší peřeje, trochu těžší a svižnější než Imster schlucht, ale všechno na oči. Celkově je to všechno o trochu větší, má to sílu, ale na pohodu. 30 min.

\info[Pitzbach]{23.8}{45cm Wenns}
\usek[most Wiese 47.115556, 10.795853]{47.129180, 10.768366}

U vysedačky je plácek u řeky, kde sme spali. Nasedali sme před nejtěžším úsek u mostu ve Wiese, ale dá se jezdit i o 2km vejš. Stav 45 cm už je spíš nižší, hodně se berou šutry a nedá se jim moc vyhnout, nicméně na rozumný sjetí to stačí. 3-5 cm navíc by byly asi optimální. Vhodný předem prohlídnout kvůli stromům, protože v některejch místech by se zastavovalo hodně těžko.

Nejtěžší místo je hned za nasedačkou, 300m peřej v pravý zatáče, najíždí se vlevo, potom je nutný dostat se napravo na 1m drop, odkud se pokračuje víceméně středem s vodou přes několik válců. Dalších 300m je klidnějších, potom je těžší místo v levý zatáčce, nejdřív dva menší stupně, potom cca 2m stupeň, za kterym je šupna u levýho břehu, docela to frčí.Potom následuje rozlitější jednodušší úsek, až před druhou vesnici, kde je místo se dvěma skokama (viditelný ze silnice), nejlepší jet vlevo. Hned za tim se buduje navigace, takže pozor na bordel ze stavby. Ve vesnici je 2m kolmej jez, vhodný prohlídnout předem ze silnice. Potom opět rozlitější úsek, až před vysedačkou to nabírá opět spád, končí to cca 100m dlouhou těžší peřejí s několika válcema, kterou je vhodný prohlídnout. Potom už je vysedačka, případn je možný jet o 100m dál, nicméně je nutný vylezt před mostem, za nim následuje soutěska.
  
\info[Otz]{23.8}{208cm Tumpen}
\usek{Horní}

Pěkná pohodová voda, je to všechno zalitný, neni se nutný vyhejbat šutrům, ale neobjevujou se ani žádný velký válce. Všechno na pohodu na oči. 25min.

\info[Venter Ache]{24.8}{206cm Tumpen}
\usek{Spodní}

Nasedačka je most, u odbočky z hlavní je směrovka na parkoviště P4. Supr střední stav, všechno na pohodu, nic se nemusí drhnout přes šutry, ale nebyl to ani velkej letní stav s brutálníma válcema... První těžší místo je rozdělovák, jede se vpravo, kde to žene na skálu a dělá to takovej polštář, ideální lajna je co nejblíž šutru vlevo. Nejtěžší místo je 100m peřej mezi druhym a třetim mostem (odbočka u autobusový zastávky Bodenegg), nicméně jede se celý s vodou, v nájezdu dva válce. Pod mostem je blbý místo vlevo u skály. Těžší místo je pravá zatáčka, kde se najíždí proti skále  a dělá se tam boční válec. Nájezd kdekoliv vpravo od středu. Ještě je tam několik těžších míst, ale všude se dá v klidu zastavit a prohlídnout. První jízda s prohlíženim 1h, druhá jízda 30min.

\cara
Následuje přejezd na Isel, jedeme ze Söldenu proti proudu, což je cesta na pass 2500m (platí se 14eur), kde se přejíždí do Itálie. Nicméně v Itálii se sjede do 720m a pak se jede přes další pass 2400m, ze kterýho se sjede na cestu z Brenneru. Kus cesty po brennerstrasse, potom doleva přímo na Defereggenbach z druhý strany, což je opět přes pass cca 2300m. Ten je navíc jednosměrnej, v tomhle směru pouštěj auta mezi 30. a 45. minutou každou hodinu (semafor je u jezera). Tahle cesta opravdu neni optimální, lépe se vrátit a jet přes Brenner, nebo to objet celý přes Rakousko.

\info[Defereggenbach]{25.8.}{62cm Hopfgarten}
\usek{Standardstrecke}

I za tohodle stavu pořád eště v pohodě jetelný a vůbec né nudný. Bohužel na tunel nebo wasserfallstrecke je to pořád moc.

\info[Tauernbach]{25.8.}{160cm Prossegg}
\usek{Standardstrecke}

Střední voda, ale sneslo by to na pohodu i o 5 - 10cm víc. Nasedačka cca 200m nad dřevěnym mostem u vysokýho bočního vodopádu. První kilák v pohodě, potom cca 800m těžší úsek WW4-5. První místo je taková úzká škvíra s bočnim válcem, určitě prohlídnout. Jedu jako jedinej, nájezd z pravýho středu do leva, ve výsledku  průjezd úplně v pohodě. Hned potom následuje hodně kamenitá peřej v levý zatáčce, jede se středem s vodou. Potom několik peřejí a stupňů, dá se jet na oči. Poslední prudší stupeň vhodný jet spíš vpravo. Poslední těžší místo je 1m stupeň pod mostem, vhodný prohlídnout z levýho břehu nebo předem z cesty. Vysedačka kdekoliv za mostem, po cca 400m vožný lesem vylezt na parkoviště u odbočky z hlavní silnice. Nutný zastavit tady, následuje těžkej kaňon zapadanej stromama.

\info[Isel]{25.8.}{280cm Lienz}
\usek{od kataraktu do Hubenu}

Spíme u řeky pod kataraktem na levym břehu, asfaltová odbočka u vesnice Feld (46.956511, 12.560980). Nasedáme těsně pod kataraktem. První peřeje trochu těžší, WW4, potom se to trochu zklidňuje. Za tohodle stavu na pohodu poježdění, na vyblbnutí, žádný nebezpečný místo. Hořejšek je čerstvě převoranej, zmizely velký kusy břehů a řeka místy teče úplně jinudy než dřív. Až na plážičku pod Hubenem dáváme za 23min. Jedeme 3 jízdy během dvou dní...

\akce{Lipno}
\datum[30.8. - 31.8.]

Letos to byly opravdu vydařený Čerťáky :) Teklo 20 kubíků, dali sme 3 jízdy v pátek, 2 v sobotu, kdy sem si rozbil hlavu a bylo hotovo... Mezitim eště Kuba zvládnul zlomit druhý pádlo za sezónu a Vašek zlámal taky jedno... Asi ani neni třeba rozepisovat další detaily...

\akce{Podzimní Ötz}
\datum[11.10 - 12.10]

Víkendová voda se Štěpánem, Pavlem a Vojtou. V pátek ve 3 ráno dorážíme na parkoviště do Haimingu, kde spíme. Ráno vyrážíme rovnou na lesní. Druhej den spíme na mezimostí, kde potkáváme partu plzeňáků, poměrně slušná párty :)

\info[Otz]{11.10.}{180cm Tumpen}
\usek{Lesní}

Dali sme za den 3$\times$, dost nízkej stav, ale eště rozumně jetelný.

\info[Venter]{11.10.}{180cm Tumpen}
\usek{Dolní}

2 rychlý jízdy (na poslední sme nasedali v půl šestý), samozřejmě taky na spodní hranici, ale dalo se.

\info[Otz]{11.10.}{178cm Tumpen}
\usek{Mezimostí}

Přidal sem se na jednu jízdu k týpkům z Plzně. Stav docela nízkej, za moc nižší vody bych to jet nechtěl, nicméně rozhodně to bylo s vodou lepší než lesní nebo Venter a dalo by se to i za nižší (možná až do 170?).

Vstupenku bylo třeba najet úplně zleva a pak klasicky pravou lajnu. Pod vstupenkou je pořád strom, musí se kolem něj úplně doprava. Potom celý tak nějak pravej střed až do zatáčky. Tam neni moc co vymejšlet, jede se středem s vodou. Potom před prvnim mostem to chce spíš doleva, vpravo je to hodně šutroviště. Jez jede jenom jeden týpek, je to samej šutr, za moc to nestojí.  100m pod jezem to najíždíme napravo, potom se jede středem až ke stupni uprostřed. Peřej nad druhym mostem najíždíme úplně zleva, pak se to jede tak nějak s vodou, za tohodle stavu se tam nedá nic moc vymyslet. Pod mostem eště 200m všechno vpravo, potom se to docela rozlejvá a je to spíš o vyhybání šutrům. Cca v půlce tohodle poledního úseku v mírný pravý zatáčce větší peřej, ke konci hodně kamenitej stupeň bez žádný jednoznačný lajny, pod nim jet spíš vlevo. Potom až ke konci k jezu v klidu.

Jez na konci je cca 3.5m, nejde přes něj moc vody, ale jedem vpravo. Úplně u pravýho břehu to padá na šutr, lepší jet to kolem středu. Hloubkou bych si taky nebyl úplně jistej, takže radši naboofnout. Všichni v pohodě, pod mostem končíme. S lehkym prohlíženim vstupenky, přenášením jezu v půlce a lovením random pádla u spacího místa jedeme cca 1h.


\rok{2015}
\akce{Čenkárna}
\datum[10.1.]

\info[Otava]{10.1.}{163cm Rejštejn (2. SPA)}
\usek{Čenkárna}

Brzká obleva, bylo kolem $15^{\circ}C$, takže paráda. Teklo cca $100m^3/s$. Byly tam akorát velký vlny a rychlej proud, všechno hodně na pohodu. Jedinej problém byla hodně studená voda a obří klády, který občas plavaly v řece. Jeli sme to $2\times$, každá jízda cca 30 min. Bylo nás celkem 7 - Pat, Ježura, a nějaký další lidi od nich z Prahy. Rozhodně jeden z lepších lednovejch víkendů :)


\akce{Jizerky}
\datum[10.3. - 12.3]
Štěpán, Slávoš, Dan,a další

\info[Jizera]{10.3.}{$15m^3/s$ Jablonec}
\usek{Z Mejta}

Jedna jízda se Štěpánem, no big water to nebyl :)

\info[Mumlava + Jizera]{11.3.}{$<3m^3/s, 16m^3/s$}
\usek{Klasika}

Taky hodně spodní limit, ale sešoupat to šlo, pod jezem bez vody

\info[Jizera]{12.3.}{$19m^3/s$ Jablonec}
\usek{Z Mejta}

Už to bylo jetelný, spodní limit na rozumnou jízdu

\info[Mumlava]{12.3.}{$4m^3/s$}
\usek{Klasika}

Trochu lepší než v pátek, dalo se to sešoupat i pod jezem.

\info[Jizera]{12.3.}{$19m^3/s$ Jablonec}
\usek{Od železničního mostu}

Hodně velká šutrovačka. Spodní limit je prej tak 25 kubíků, za 35 je to prej hezký. V poslední větší peřeji před Cutisinem mě to namáčklo na šutr uprostřed řeky a valilo se na mě celkem dost vody ze shora, takže sem neměl šanci nic udělat. Naštěstí Slávoš byl docela rychle z lodi a vytáh mě na házečce. Pak sme jeli Cutisin, moc vody to němlo, ale v pohodě. Příště se tam snad podívám za lepších stavů...


\akce{Kamenice}
\datum[18.3.]
Štěpán, Mudra, Jirka

\info[Kamenice]{12.3.}{$7m^3/s$}
\usek{Horní}

Koukali sme na ten úsek pod přehradou, ale je to celkem mordor. Hlavně ta zatáčka s nejtěžšim místem. Jako rozjezd na začátku sezóny to pro mně fakt nebylo. Krom tý nejtěžší části je to takovej Pitzbach a dá se to, ale chce to nakoukat stromy. Sešoupli sme toho jenom kousek pod nejtěžšíma místama a jeli sme horní až do Tanvaldu. Horní je celkem o ničem, jsou tam nějaký jezy a pár peřejí, ale nic moc. 2 jezy se přenáší.

\info[Kamenice]{12.3.}{$12m^3/s$}
\usek{Navarov}

Odpoledne sme 2x šoupli Navarov, druhá jízda už jenom na zbytcích vody, ale je to hezký poježdění.



\akce{Piemont a Tessin}
\datum[1.5. - 10.5.]
Kuba, Pat, Šuhy, Boleslaváci

Vyrážíme ve čtvrtek večer z Prahy, ale nějak to nestíháme, tak se rozhodneme přespat v Budějcích. Vzhledem k tomu, že sme nákup v Tescu dokončili nějak po půlnoci a komplet nabalný sme byli asi v půl druhý, tak stejně nemělo moc smysl už vyrážet. Ráno v 7 vyrážíme směrem Ötz. Celou cestu leje, ale na Ötzu se trochu vyčasí. Dorážíme tak odpoledne, ale docela v rozumnej čas. Koukáme na Mezimostí, ale za 175cm je to úplně suchý, tak jedeme dolu.

\info[Otz]{1.5.}{$175cm Tumpen$}
\usek{Wellerbrücke}

Dáváme 3 jízdy, lajny sou docela jasný, největší problém dělá válec na středu TNT, kde si v první jízdě trochu  pokempim, ale dá se z toho vytáhnout. Za tohodle stavu to nemá moc sílu, takže sice je to pořád stejně z kopce, ale na všechno je dost času. Druhá jízda podobná, ale celkově to dáváme docela v pohodě. Při třetí jízdě eskymuju pod champions killerem, ale naštěstí až za nim. Kuba dává poslední lajnu přes killer úplně parádní. Ale už to nechcem pokoušet a jedem dolu, ještě nás čeká přejezd do Tessinu.

Večer přejíždíme na spaní k Verzasce, spíme na parkovišti ve vesnici cca. 500m nad nasedačkou středního úseku. Platí se 20 franků / eur za noc+den, z toho 10 franků se platit musí stejně, protože všechny parkoviště v údolí jsou přes den zpoplatněný. Noční lístek se nám ale už nepodařilo zaplatit, asi už bylo pozdě či co. Takže kupujeme až ráno lístek jenom na parkování.

\info[Verzasca]{2.5.}{$24 m^3/s$ Lavaretezzo}
\usek{střední}

Ráno je parádní počasí, sluníčko, teplo, začínáme na střední Verzasce pod mostem. První větší peřej kousek po nasedačce prohlížíme, je to dost kamenitý (na 40 kubíků je tam lajna mnohem lepší). Nakonec jedeme průskokem ze středu doleva a pak úplně kolem levýho břehu, případně se dá levej střed přes šikmej válec. Druhou těžší peřej taky prohlížíme, jede se celá levej střed a ke konci spíš doleva. V nájezdu eskymuju, ale naštěstí zvedám hned na pohodu (celkově na začátku zájezdu eskymuju zbytečně skoro na každý řece, ta nerozježděnost je přece jenom znát). Pak už je to většinou na oči, maximálně s rychlim prohlížením, kdy jeden vyběhne z lodi. Přece jenom lajny jsou úplně jiný, než jak si je pamatuju za 40 kubíků. Těžší místa jsou až ke konci, kde jsou dva skoky za sebou a pak ďáblova strouha. První skok jedeme zprava, druhej se jee spíš u levý strany. Strouhu najíždíme úplně zprava, roládou se nechávám hodit úplně k pravýmu břehu a pak celý vpravo, na konci peřeje je docela válec, člověk tam chce bejt co nejvíc vpravo to de.

My tady končíme, většina Boleslaváků jede až dolu, ale klukům se nechce. Nakonec prej většinu těch klasických nejtěžších peřejí přenesli, ale prej to bylo supr. My dáváme místo toho ještě druhou jízdu. Už je trochu míň vody, podle vodočtu $22m^3/s$ a je to znát, ale zas už to máme najetý, takže to sjedem za 45 min.

Po jízdě se deme podívat na začátek spodku - k vodopádu a pak do vesnice. Přece jenom jako první jízdu bych to asi chtěl jet za poloviční vody, ale ten spodek už je pak asi rozumnější. Ještě se stavíme na přehradě a potom vyrážíme na Rovanu, kam jedem za Ježurou (je tam s Djembeesem, Ivošem, ..), od kterejch primárně potřebujeme půjčit vařič :) Na Rovaně super spací místo pod vesničkou Niva na louce u nasedačky ($N46^\circ17.2956' E8^\circ31.8210$).


\info[Rovana]{3.5.}{$12-14 m^3/s$ odhadem}

Ráno nasedáme na Rovanu, tušíme, že to má docela nakoupeno (bežnej stav $4-10m^3/s$), ale vypadá to rozumně. Slezeme k vodopádu, dojedeme na soutok a po asi 200m prohlížíme první peřej, která je zakončená hodně velkym válcem, kterej nejde moc rozumně najet (bylo by to přinejlepšim 50:50). Dojíždíme  těsně před válec do tišiny vpravo, přenášíme asi 50m. Potom jedeme eště asi 200m, kde sou 3 takový stupně. Opět prohlížíme, další peřej končí opět dost podobnym válcem, proud navíc naráží na protilehlou stěnu a vrací se to docela z dálky. Já bych do toho asi šel, ale nakonec se shodujeme, že to asi nemá cenu a vzdáváme to. Vylejzáme na levym břehu na stezku a vracíme se k autu (cca 20 min cesta s lodí). 

 Kluci, co ráno nasedli na Rovanu mnohem vejš, tam zůstali někde v nějakym kaňonu a museli vylejzat na házečkách ven, tam nakonec stráví několik hodin a dostávaj se ven asi ve 3, naštěstí prej všichni v pohodě. My přejíždíme odpoedne na Ribo. 


\info[Ribo]{3.5.}{vyšší střední}
\usek{Od Vodopádu}

Nasedáme pod vodopádem, nějak na něj za tohodle stavu nemáme morál - den před tim tam byly nějaký zlomený žebra, pádla, někdo si tam udělal něco s obratlem, což nás od toho docela odradilo. Jinak je to moc hezký, má to dobrou vodu, přenášíme jenom jedno místo - první větší peřej (300m pod vodopádem), kde je prej vlevo pod skálou na kterou to žene undercut. Za tohdle stavu je to celý pod vodou, ale asi to nemá cenu. Jinak docela prohlížíme, ale všechno jedem celkem na pohodu. Jedeme cca 2 hodiny.

Večer přejíždíme na spací místo pár km na Vergelletem, kde je vyjetá cesta na louku přímo u řeky, spaní parádní.

\info[Ribo]{4.5.}{vyšší střední}
\usek{Od vodopádu}
Ráno dáváme další jízdu. Koukáme i na horní úsek, kterj vypadá dobře, ale nakonec se nám nechce jet kaňon na začátku vesnice. Ale podle videí je to asi docela v pohodě, určitě to příště musíme zkusit. Jinak druhou jízdu jedeme asi 40min, už se nezdržujeme prohlížením, všechno najíždíme na pohodu. Jedeme jenom 3 bez Pata, tomu se nechce a převáží auto.

\info[San Giovanni]{4.5.}{nižší stav, $1.5 - 2m^3/s$}
\usek{Park'n'huck ($N46.007642, E8.584774$)}
Rozhodujeme se přejet do Piemontu na Sesii, tak se cestou stavíme na San Giovanni. Vody moc neni, ale na vodopády je to v pohodě. Dáváme si tady na pohodu 3 kolečka, až na jednu Patovu lajnu, kdy zvedá pod prvnim dropem a druhej jede pozadu. Ale všechno v pohodě. Poslední jízdu dávám na třetím dropu freewheel, kupodivu se docela daří, tak to vyhecuje Kubu a taky tam jednoho parádního pošle.

3 jízdy nám docela stačí, k večeru přejíždíme na Sesii. Na spaní se připojujeme k Boleslavákům, spíme u pěšího mostu ve vesnici Scopa, je tu parádní plácek s lavičkama ($N45^\circ47.0148' E8^\circ6.5166'$).


\info[Gronda + Sorba]{5.5.}{vyšší střední}
\usek{klasika}

Vynášíme si to asi 500m, nahazujeme to pod sifonem na Grondě. Jedeme těch pár klasickejch dropů, hodně fotíme a točíme, lajny super, paráda. Na dropu pod mostem Pat lajnu moc netrefuje, padá bokem, ale vyzvedá. Pokračujem k mostu, cestou nás překvapuje jeden válec, ve kterym na chvíli pokempim, potom tam i na chvíli zůstává Pat. Sice se z něj dostává, ale necejtí se na spodek a končí. Potom jedeme po jednom slidy, jistíme válec pod třetim a fotíme. Všichni jedou na pohodu. Pokračujeme dál, za malou skluzavkou s válcem je strom přes celou řeku i s větvema, jedem hodně vpravo. V další rozbitý peřeji je v nájezdu strom, přenášíme půlku zleva, potom nějak pochybně nasedáme, asi by bylo lepší to jet celý. Potom klasická peřej drop přes kamen vpravo a kaňonek, Kuba se Šuhym se lehce diví, že to nevypadá tak jednoduše, jak sem řikal :), ale dávaj na pohodu. Já natáčim, jedu poslední. Potom klasicky pár menších peřejí, prohlížíme až drop co se jede z prava, potom ostrá pravá zatáčka. Jedem všichni na pohodu, až na to, že pod touhle peřejí otevřu futrál na GoPro, abych ho odmlžil. Jenže na to nějak zapomenu, vemu hlemu do ruky a kamera skončí ve vodě. Naštěstí rychle vyndám baterku a do druhýho dne sušim, kupodivu přežila. Potom je dvojtej drop s válcema, vlevo je undercut s lehkym sifonem, pro jistotu přenášíme, i když by se to dalo jet. Potom je klasickej metrovej drop, je tam slušnej válec, ale proud navádí přímo na optimální lajnu pro boof, jedem všichni na pohodu. Potom úsek v pohodě, další větší peřej jedeme těžkou levou lajnu kolem skály. Další peřeje jedeme na oči, až k poslední peřejí před mostem, kterou nejedeme. V nájezdu je šutr a je to hodně úzkej průjezd mezi šutrama, potom už by to šlo, ale nestojí to za to.

Přejíždíme na Mastallone, spíme na parkovišti u kapličky nad Landwasserem.

\info[Mastallone]{6.5.}{vyší střední}
\usek{spodní}

Nasedáme kousek nad mixérem na parkovišti ($N45.854696^\circ~E8.231881^\circ$). Na mixéru dáváme pár koleček, fotíme a točíme, všechno v pohodě. Potom je pár peřejí 3-4, všechno jedeme na oči, až ke vstupence do kaňonu. Tady prohlížíme a jistíme, s Kubou se dohodnem, že to pojedem, já jedu první. Důležitý je trefit pravou lajnu u rozdělováku, vlevo je sifon. Válci dole už nevěnuju takovou pozornost, příde mi, že v tý rychlosti by neměl bejt problém ho přejet. Pravou lajnu sice trefuju, ale před poslednim válcem mě to rozhodí, eště to stihnu najet hodně na pravo po skále, ale už nestíhám srovnat náklon a do válce padám bokem a zůstávám v něm. Po třech eskymácích to docela rychle vzdávám, moc se toho vymyslet nedá. Šuhy mě naštěstí na první pokus trefuje házečkou a hned sem z vody venku. Pádlo odlovim hned, pro loď musim skočit na žabáka. Nakonec z vracáku odlovíme házečku a jednu pěnu ze špičky, o druhou sem přišel. Časem nacházíme i loďák s lékárničkou. 

Zbytek kaňonu je poměrně v pohodě, ale těžší, než se zdá seshora ze silnice. První peřej se jede celá vpravo, druhá peřej sou tři válce, nejlehčí lajna je všechno vpravo. První válec je zadarmo, druhej se dá po stranách, ve středu ho člověk jet nechce. Poslední je lepší jet vpravo, ale dá se kdekoliv. Potom už je jen volej. Zajímavý místa jsou až pod jezem, vlevo je sifon přes půl řeky, jedeme vpravo. Ke konci je solidní kaňon, prohlížíme, ale většinu přenášíme, v nájezdu je šutr. Vystupujeme ve Varallu za mostama vlevo (dobrej výstup po schodech).


\info[Sesia]{6.5.}{$115 cm$, vysoký}
\usek[Orto]{Mollia}

Odpoledne jedeme horní Sesii, úsek neznáme, ale je hezky a horko, takže to z toho ledovce docela teče. Úsek je asi 2.5km, má to charakter trochu jako Standardstrecke na Defe, ale má to víc vody a je to celkově větší. Prostě kontinuální WW4+, dá se to jet částečně na oči, ale hodně prohlížíme. Cca v půlce přenášíme 200m úsek, je to jeden drop přes celou řeku se solidnim válcem, po 200m je další drop kterej se dá jet vlevo, ale vpravo bejt nechceš. Nájezd do levý lajny je ale zablokovanej stromama. Konec je trochu ostřejší, prohlížíme, ale všechno se dá jet. Končí se ve vesnici u druhý lávky pro pěší, vysedačka se nesmí minout, kousek za vysedačkou je katarakt/vodopád, kterej jet nechceš. 

Večer spíme opět u Sesie u laviček.

\info[Anza]{7.5.}{odhadem 8 kubíků, nižší}
\usek[Pod velkym jezem / přehradou]{Kovová lávka pro pěší}

Krajinovka, na začátku pár těžších peřejí, ke konci lehčí. WW 3-4. Uprostřed je relativně nová peřej, napadaná hromada šutrů v řece, lajna tam za týhle vody spíš neni, přenášíme. Na konci se lodě nesou hodně do kopce (20 min) nahoru na silnici.

Co jsme zjistili až doma z videí, že se to dá jet ještě dál, což by měl bejt zdaleka nejlepší úsek (cca o stupeň těžší), nějaký must-runy a několik 1-3m vyskojech dropů. Příště to určitě musíme zkusit. Výstup by měl bejt u starýho římskýho mostu, kterej podle videí nejde minout.


\info[San Bernardino]{7.5.}{střední, odhadem 12-15 kubíků, cca 10-12cm dole na koruně jezu}
\usek{Horní}

Na San Bernardino je supr cesta pro jedno auto, neprojedou tam auta vyšší než 270cm. Na vodu se nasedá z cesty na pravym břehu pod mostem. Dorážíme trochu pozdě, nasedáme na vodu až v půl šestý, nasedáme až pod první peřejí, kde je solidní sifon. Úsek má cca 4km, jedeme s prohlížením 2.5 hodiny.

Na úseku je v podstatě 5 větších peřejí, všechny jsou v první půlce úseku úseku. První dvě menší peřeje sou v pohodě, druhou peřej najíždíme z vracáku vlevo a končí se průjezdem vpravo podél skály. První větší peřej se najíždí průskokem vlevo, potom další skok vlevo, konec se jede taky hodně vlevo, vpravo je undercut pod skálou. Druhou peřej najíždíme ze středu, potom průskokem vpravo proti válci, kterej de zleva. Levá lajna by taky bala možná. Po 50m chytáme tišinu vpravo a nejprudší část přenášíme zprava. Lajna tam je, ale peřej je dost blbá a těžká, končí takovym skokem do válce přímo proti skále. Třetí peřej najíždíme ze středu přes šutr a potom se musí hodně doleva (vpravo je sifon), na konci je taková skluzavka u levý strany, která končí válcem, kterej je za větší vody prej docela solidní, ale za týhle vody žádnej problém neni. Čtvrtá peřej je levotočivá zatáčka, najíždíme hodně zprava, první válec středem, druhej větší skok jedeme přes placák hodně vlevo, vpravo to de mezi šutry. U konce je velkej šutr uprostřed, možný jet z obou stran. Potom je menší peřej, jede se pravej střed. Poslední větší peřej je must run najíždíme zleva, za velkym šutrem chytáme pravej vracáck uprostřed, poslední část jedeme středem.

Ke konci řeka zvolní, peřeje jdou už jet na oči nebo zkouknout z lodi. Vyčnívá jedna peřej s větším válcem, jedeme ze středu přes šikmej válec hodně do prava podél skály, válec vlevo je poměrně solidní. Potom už zbejvá jenom pár peřejí, končí se u nesjízdnýho jezu na pravym břehu, kde je nutný vynést lodě po schodech do kopce. Řeka je to fakt super, určitě se sem musim zas někdy podívat. Má to neskuečně čistou vodu, je vidět i do víc jak 10m.

Večer přejíždíme na nasedačku Melezzy, kde spíme na parkovišti u mostu ($N46.137942^\circ, E8.573761^\circ$).


\info[Melezza]{8.5.}{nízký stav, odhadem $8 - 10m^3/s$}
\usek[Pod mostem Melezzo 3]{Jez s lávkou}

Nasedáme nezvykle brzo dopoledne, máme před sebou úsek asi 3km a nikdo to neznáme. Nasedáme až pod mostem, protože v peřeji nad mostem je novej sifon a před pár tejdny se tam někdo utopil. Prvních pár set metrů rozbitý peřeje, máme spodní stav, aby to šlo vůbec projet. Potom se to trochu rozjíždí, sou tam dvě větší peřeje, jsou hodně kamenitý, ale dá se to projet, WW4. 

Potom se objevuje kaňon s kolmýma stěnama, první velká peřej je asi nejtěžší na celym úseku. Za týhle vody je první skok nejetelnej, vlevo padá do sifonu a vpravo do undercutu. Naštěstí se dá zastavit těsně před tim vlevo, za větší vody ale bude to zastavení hodně prekérní. Pokud bych to musel jet, tak najet ze středu doprava. Já nasedám pod tim a jedu následující 2m drop úplně zprava po skále, potom rychle následuje úzká pasáž se 2ma válacema. Za větší vody by se dal jet drop zprava i zleva. Kuba se Šuhym se na to moc netváří a dávaj alpine start ze šutru uprostřed, protože člověk nahoře sám moc dobře nenastoupí a přenest se to nedá.

Hned další peřej je cca 2,5m drop, obtížně se prohlíží i přenáší. Nejvíc vody teče vlevo, ale úplně vlevo většina vody mizí pod skálou. Najíždíme to levou stopou, ale ze středu spíš do prava, vlevo je kapsa. Za vyšší vody zde budou pravděpodobně 3 lajny, vlevo, uprostřed a hodně vpravo. Volil bych tu pravou nebo pravej sřed, oboje by pravděpodobně byl hodně čistej drop.

Za tímhle úsekem (cca 200m) se řeka zklidňuje, peřeje už jsou max WW4, některý je ale třeba prohlídnout, charakter není moc přehledný. Vyniká peřej s malym slidem, kterej se jede ze středu do leva, vpravo to de dost do šutru. Hned následující peřej se jede zprava doleva, vpravo to de pod skálu.  Asi uprostřed úseku přenášíme jednu peřej, je hodně rozbitá a za týhle vody jetelná jenom těžce přes šutry, navíc vpravo je pod skálou docela kapsa. Charakter už se příliš nemění, dojíždíme k jezu s ocelovou lávkou přes řeku, nejde přehlédnout. Jedeme víc jak 3 hodiny. Následuje výšlap 10 min do kopce. Vyleze se přímo za hranicí (nasedá se v Itálii, vysedá ve Švýcarsku), pro jistotu bereme s sebou občanky. Mohlo by se jet asi 500m dál až do vzdutí přehrady, kde je most (odbočka na Moneto), vyhlo by se nošení lodí do kopce.


\info[Moesa]{8.5.}{vyší střední $60m^3/s$}
\usek[Sorte]{Cama (3. most po cestě)}

Na večer dáváme na vypádlování rychlou jízdu, sesmahnem to asi za 30min. Teče to hezky, před klasiskym průskokem se nedá zastavit, jedeme na oči. Vlevo průjezd v pohodě. Jinak to všechno docela teče, ale je to poměrně na pohodu. Rozhodně to stojí za sjetí.


\info[Glenner]{9.5.}{$20m^3/s$}
\usek{klasika}

Nasedáme ve vesnici nad jezem a jedeme před jez na soutoku. Nad jezem je jenom jedna větší peřej kousek za soutokem Glenneru s Valserrheinem. Jez jedeme na pohodu levej střed. Za tohdle stavu je řeka docela kontinuální a zábavná, od soutěsky pod mostem je hodně pěknej cca 1km kontinuální úsek až na konec galerie.
Jedeme 2 jízdy, obě pod hodinu.

Večer přejíždíme na Lammer, spaní na nasedačce je ale plný (WW ženy), tak nakonec spíme pod mostem na vysedačce. Přes noc je ale brutální bouřka a Lammer stoupne z 80cm na 105cm. Ráno to procházíme, lajny tam sice sou, ale vody je fakt hodně a nakonec to nejedeme a zkoušíme Taugl.


\info[Taugl]{10.5.}{vyšší}
\usek{vodopád}

Ráno dáváme 2 jízdy na vodopádu na Tauglu, je trochu vyšší voda, přes levou část vodopádu jde trochu vody. První jízda super, docela trefim boof a dávám to v pohodě. Druhou jízdu moc netrefuju, dole eskymuju a končim ve vracáku vlevo hned u vodopádu. Naštěstí se z toho dá v pohodě vyjet. Vodopád je hodně zábavnej a je o trochu vyšší, než se zdá ze břehu.



\akce{Inny a okolí}
\datum[20.6. - 22.6.]
Víťa, Mišu, Alča, Vojta, Vašek, Plukovník, Lovec

\info[Inn]{20.6.}{$260cm$ - Kajetansbrücke}
\usek{Tössener schlucht}

Pohodička na rozpádlování, teče to, ale vody by mohlo bejt víc.


\info[Inn]{20.6.}{$42m^3/s$ Tarasp}
\usek{Giarsun + Ardez}

Má to moc hezkou vodu, asi takovej skoro optimální pohodovej stav. Jedem všechno na oči, krom řezníka, kterýho přenášíme celýho. Dá se to jet cca do půlky, ale docela to tam teče a dá to trochu zabrat, dostat se k levýmu břehu. Jinak je to všechno v pohodě, krom peřeje hned za řezníkem, kde ta levá lajna neni úplně optimální, jede se tam přes velký vejvary a já tam zvedám. Většina jede lajnu u pravý strany, kde se to dá jet asi přes 1m boof.


\info[Inn]{21.6.}{$28m^3/s$ Tarasp}
\usek{Giarsun + Ardez}

Jedeme druhou jízdu, je znatelně míň vody, ale pořád je to pohodový. Tentokrát jedeme celou peřej před řezníkem, cca do půlky přenášky, kterou to výrazně urychlí. Jedeme jenom já, Víťa a Vašek, všichni zastavujem v pohodě, ale v posledním stupni to chce jet spíš přes kicker směrem doleva, ostatní jedou víc na střed a pak musí k levýmu břehu docela máknout.


\info[Pitzbach]{21.6.}{$53cm$ Wenns}
\usek{Standardstrecke}

Pitzbach má super stav, asi skoro optimální. Bohužel už je hotovej novej jez, kterej si značnou část vody veme. Pod jezem to odpovídá tak stavu 40cm, takže se to dá dojet, ale je to velká šutrovačka. Nicméně hořejšek je OK, já teda najíždim zatáčku dost bídně a dole zvedám, opět mě zvrhá šutr u pravý strany.  Zbytek docela v pohodě. Potom za peřerjí s jazykem vlevo sou stromy, musí se jet úplně u pravýho břehu. Potom jedem klasickej skok, všichni na pohodu, za tohodle stavu je to fakt super. Za zatáčkou pod skokem na nás vykouknou 2 velký stromy spadlý přes řeku, naštěstí jsou vidět včas, vklidu zastavujeme a přenášíme. Potom už se jede na pohodu až k jezu, poslední peřej před jezem je už trochu zalitá, dá se jet na obou stranách od rozdělováku. Jez přenášíme vpravo. Pak už je to bída.


\info[Venter Ache]{21.6.}{$215cm$ Tumpen}
\usek{Spodní}

Venter za tohodle stavu je super, vůbec to eště neni těžký, ale teče to kontinuálně, neni potřeba se vyhejbat šutrům. Lajny sou ale v podstatě stejný jako za menších stavů, všechno v pohodě jetelný.


\info[Otz]{21.6.}{$215cm$ Tumpen}
\usek{lesní}

Ještě stíháme jednu rychlou lesní, za týhle vody už to má docela sílu, trochu mě to překvapuje. Přijde mi to o trochu těžší, než ten spodní Venter. Vašek si tam asi 1km od startu zaplave, naštěstí docela v dobrym místě, rychle se dostává ke břehu a hned chytáme i loď. Jinak jedeme bez větších problémů, je to fakt rychlovka, jedem asi půl hodiny.


\akce{Isel a Itálie}
\datum[3.8. - 7.8.]
Kuba, Štěpán, Pavel

V Praze máme dobrej start, v Budějcích přesedáme do našeho auta a v jednu už sme na kempovišti na Iselu pod kataraktem.

\info[Isel]{3.8.}{$300cm$ Lienz}
\usek{pod kataraktem}

Jedeme 3x s pendlem.


\info[Isel]{3.8.}{$112cm$ Hinterbichl}
\usek{Horní}

Jedeme 2x s pendlem

\info[Isel]{4.8.}{$290cm$ Lienz}
\usek{pod kataraktem}

Jedeme 4x s pendlem.

Prohlížíme tunel, ale 56 je moc. Koukáme i na wasserfal strecke, ale taky moc.

\info[Defereggenbach]{4.8.}{$58cm$ Hopfgarten}
\usek{pod kataraktem}

Jedeme 2x s pendlem.

\info[Ahrnbach]{5.8.}{$33.4m^3/s$ St. Georgen / $105cm$ vodočet na vysedačce}
\usek{Schloss}

Prohlížíme Reinbach, ale příde nám dost vody, v nájezdu je docela solidní válec.

\info[Rienz]{6.8.}{placák pod přehradou na úrovni hladiny ??}
\usek{Rienzschlucht}

Asi minimální průtok z přehrady.... Kluci jedou až do Brixenu, jedou 3.5h, s tim, že hodinu strávíme u nejtěžšího místa. K tomu dorazili po necelý hodině od přehrady, cca 1km.


\info[Eisack]{6.8.}{$??$}
\usek{todo}

Poslední den jedeme jernom lammer, salzach má nakoupeno (145 je fakt moc) a Taugl je suchej.


\clearpage
\akce{Rakousko, Švýcarsko}
\datum[11.8. - 19.8.]


Kuba, Šuhy, Sosan, Ježura, Dominika


Vyjíždíme až v sobotu, protože kluci přijeli do Budějc až po půlnoci. Kousek za hranicema pícháme pneumatiku, takže super start. Pozdějc v Loferu zjišťujeme, že i druhá zadní pneumatika je pekelně sjetá (na vnitřní straně čouhaj dráty), takže máme strach s tim někam dál jet, protože už nemáme rezervu. Naštěstí v neděli seženeme nějakýho týpka, co nám pneumatiku vymění, dává nám nějakou starou ojetou, ale na dojezd to snad bude stačit. Za práci si řekne 30E, za pneu nic. Celkově vydařenej začátek.


\info[Saalach]{11.8.}{$28cm$ Unterjettenberg}
\usek{Lofer}

Na rozjetí jedeme Lofer za nižšího pohodovýho stavu. Všechno je to na pohodu, nejtěžší část - trojkombinaci peřejí prohlížíme, ale je to docela na pohodu. V nájezdu je šikmej válec, je dobrý se dostat co nejvíc doleva. Potom je malej skok s válcem, jede se u levý strany. Další peřej je boof přes šutr u pravý strany, potom je lžíce, která se boofuje středem. Potom pokračujeme až k peřeji s pekelnym sifonem, tu jedeme úplně u pravý strany, aby jsme se sifonu v klidu vyhnuli. Nicméně druhou jízdu Honza na tu lajnu zapomíná a jede středem. Je tam malej boof přes šutr, ale neni to zas tak těžká lajna. Honza ale zazmatkuje a vyplave, jeho loď skončí přímo nad sifonem, on je naštěstí víc vpravo a proplave kolem. Loď nám dá trochu zabrat, ale nakonec jí vytáhneme skrz malou jeskyňku u břehu. Naštěstí je všechno OK.

\cara
Spíme u štěrkovny u Loferu $N47.538019^\circ~E12.726669^\circ$, protože se bojíme s autem jet někam dál. Nicméně spaní je tu super.


\info[Saalach]{12.8.}{$28cm$ Unterjettenberg}
\usek{Lofer}

Ráno jedeme ještě jednou Lofer. Je to v pohodě, dokud se Šuhy nenechá i s Axiomem natlačit pod šutr a plave. Naštěstí všechno rychle odlovíme.


\info[Otz]{12.8.}{$250cm$ Tumpen}
\usek{spodní}

Odpoledne jedeme $2\times$ nakoupenej spodní Ötz. První jízda je trochu opatrnější, nikdo to za podobnýho stavu nejel. Nicméně je to docela v klidu, jsou tam akorát velký vlny a rychlej proud, ale je to přehledný a všechno se dá objet. Před jezem se dá v klidu zastavit na pravý straně. Až do Innu je to super, dojezd po Innu je trochu nudnější, Inn je hodně zarovnanej. 

\cara
Spíme na vysedačce v Haimingu společně s budějovičákama (Radim).


\info[Otz]{13.8.}{$235cm$ Tumpen}
\usek{spodní}

Ráno dáváme ještě jednou spodní Ötz, voda je znatelně nižší, ale charakter podobnej.


\info[Inn]{13.8.}{$27m^3/s$ Tarasp}
\usek{Giarsun + Ardez}

Za tohodle stavu velká pohodička, jedeme bez problémů $2\times$. Druhou jízdu jedu jenom já s Ježurou, dáváme to za 57min.

\cara
Večer přejíždíme na Landquart, kde trochu hledáme spaní, ale nakonec spíme na vysedačce horního úseku. Řeka nemá vodu, takže ráno jedeme pryč. Cestou prohlížíme Medlsbach a Reuss, ale ani jedno nemá vodu. Reuss vypadá super, ale je tam několik přehrad, takže s průtokem je tam problém.

Cesta vede přes 2 pasy, první má něco přes 2 tisíce, potom se eště musí na Furkapass, kterej má asi 2500. Cestou zabijeme většinu dne, jak kvůli pasům, tak při prohlížení úseků, který nakonec nejedeme.


\info[Rhona]{14.8.}{$23m^3/s$ Reckingen, vyšší střední stav}
\usek[kemp Grafschaft $N46.4553836^\circ~E8.225618^\circ$]{nádraží naproti Mühlbachu $N46.411470^\circ~E8.154963^\circ$}

Řeka má docela nakoupeno, jedeme navečer a celej den bylo vedro. Začátek je pohodička na rozpádlování, cestou akorát potkáváme jeden jez s větším válcem. Jede ho jenom Ježura, ale dá se to docela v pohodě. Asi 5km je řeka docela poklidná, max WW3. Potom míjíme betonový most, za kterým je vlevo sesutej svah / štěrkovna. Tady začíná mít řeka výraznější spád, za tohodle vodního stavu WW4. Následují relativně kontinuální 3km, ale není tu nic záludnýho. Akorát je potřeba znát vysedačku, lávka, kterou uvádějí kilometráže, už na vysedačce není. Navíc u vysedačky se řeka nabírá na obtížnosti a pokračuje do kaňonu WW5, takže je potřeba zastavit včas. Kaňon lehce prohlížíme z vysokýho lanovýho mostu a vypadá to fakt hezky, ale za týhle vody je to strašně nakoupený.


\info[Lonza]{15.8.}{odhadem $15m^3/s$, vyšší střední}
\usek[nad vesnicí Willer $N46.40792^\circ~E7.795144^\circ$]{na přehradě}

Jedeme dopoledne, ale vody to má docela dost. Je to hodně kontinuální úsek bez vracáků, WW4. Nejtěžší místo je kousek za nasedačkou za betonovym mostem, kde je to trochu víc z kopce. Na tuhle řeku docela sedí popis z DKV. Jedeme to jinak všechno v pohodě, společně s přepádlováním přehrady asi 25 min.


\info[Rhona]{15.8.}{$115m^3/s$ Brig, vysoký stav}
\usek[kemp Grafschaft $N46.4553836^\circ~E8.225618^\circ$]{nádraží naproti Mühlbachu $N46.411470^\circ~E8.154963^\circ$}

Jedeme docela big water úsek na Rhoně, je pekelný vedro a jedeme odpoledne, takže to má celkem nakoupeno. Předem ze silnice prohlížíme dvě největší peřeje, ale zbytek vypadá OK. Po nasednutí je to celý trochu větší, než se zdálo ze silnice, je to big water 4+, dvě největší peřeje jsou za 5. Peřeje kolem kempu, než se dojede k mostu, jsou docela velký, místy jsou válce, který je opravdu vhodný objet. Potom za silničním mostem následuje první velká peřej, najíždíme úplně vlevo podél břehu, hned za nájezdem jedeme těžce do prava, dole je masivní válec vlevo, kterej je potřeba minout. Reálná lajna je i těsně vlevo podél břehu, kde válec pouští - tu jede Ježura. Šuhy přenáší, my jedeme na pohodu. Po pár set metrech je druhá větší peřej, najíždíme středem, potom je potřeba je těsně kolem levýho břehu přes jazyk, kterej objíždí válec, hned potom se jede do prava, minout zatopenej šutr a potom boof přes poslední válec, kterej už je docela v pohodě. Potom následujou dva takový stupně, oba jedeme u levý strany, vpravo sou mocný válce. Potom sou pořád docela mocný peřeje, je to hodně kontinuální WW4 až do vesnice Mörell / Fillet, kde se to zklidňuje a obtížnost o stupeň klesá. Na konci vesnice je jez / přehrada, přenášíme těsně nad hranou jezu u levý strany, za větší vody bych tam ale určitě nejezdil. Zbytek až na vysedačku je docela v pohodě, asi 500m před vysedačkou je stupeň s válcem, kterej je nutný naboofnout. Je to ale označený cedulí na břehu.Vysedá se na pravý straně pod silničním mostem. Jedeme asi 2 hodiny s prohlíženim a jištěním první velký peřeje.


\info[Rhona]{15.8.}{asi $250m^3/s$}
\usek[pod přehradou Leuk $N46.311472^\circ~E7.633883^\circ$]{most Sierre}

Je to fakt mocná big water záležitost. Peřej pod přehradou je WW5+, za týhle vody je první půlka peřeje úplně brutální, sou tam dva mocný válce asi 2m vysoký, potom následuje pole zalitejch šutrů, který se nedá načíst. Nasedáme až na spodku peřeje, kde sou masivní vlny, ale nic dramatickýho. Konec peřeje dojíždíme v pohodě, postupně se to zklidňuje, je to takový WW3 s velkejma vlnama, jedeme asi 4km. Na konci za mostem je super velká vlna, mít s sebou rodeovku, bylo by to fakt super.

\cara

Spíme na pěknym místě, kousek od nasedačky přes silnici $N46.308617^\circ~E7.633147^\circ$, je tu i pitná voda.


\info[Rhona]{16.8.}{$120m^3/s$ Brig}
\usek[Grengios kemp]{Natters silniční most}

Ráno dáváme druhej run. Jde to docela pěkně od ruky až k druhý peřeje, kde se Šuhy zvrhá v nájezdu, nezvedá včas a celou peřej proplave hlavou dolu, na konci krysí. Ježura ho rychle dostane na levej břeh, já se vydávám za lodí. Docela to teče a udělat s tou lodí něco kloudnýho moc nejde, nakonec se mi jí podaří cca po 15min natlačit ke břehu, kde se zachytne na kamenech. S Ježurou jí vylejváme a přetaháváme na pravej břeh, kde nás po chvíli kluci naberou a jedeme pro Šuhyho. Ten je na druhym břehu a nemůže pryč. Chceme mu na házečce přetáhnout loď, aby to dojel, ale už se mu do lodi za žádnou cenu nechce. Nakonec slanim kousek ze silnice k řece, kde se ho pokoušim přetáhnout na házečce. Neni to vůbec dobrej nápad, házečku neudržim, i když je přes smyci zaháklá na kameni, tak něco povoluje a Šuhy odplavává i s házečkou. Naštěstí je u něj Ježura, pomáhá mu dostat se ke břehu, odkud už se Šuhy nějak vyškrábe na silnici. My s Ježurou dojíždíme v klidu na vysedačku.


\info[Drance]{16.8.}{spodní limit}
\usek[Orsieres pod jezem]{Někde v půlce}

Je to hrozná jebačka, vysedáme asi v půlce. Kdyby to mělo nakoupeno, bylo by to slušný WW4, hlavně na začátku ve vesnici je to docela z kopce, je tam několik skoků do 1m.

\cara

Spíme na štěrkovišti u řeky $N46.081792^\circ~E7.099270^\circ$.

\info[Dora Baltea]{17.8.}{$40+m^3/s$ odhadem, vysoký stav}
\usek[most nad Pré-Saint-Didier $N45^\circ46.105 E6^\circ58.881$]{most Morgex $N45^\circ45.4896' E7^\circ1.7592$}

Přejíždíme na Doru Balteu. Má to na pohled docela dost vody, ale to se od ledovcovky v létě očekává. Možná je trochu zvláštní, že je slyšet, jak se po dně valí šutry, ale to nás nerozhodí. Hned na nasedačce zjišťujeme, že je to dost z kopce a celkem to valí. První peřej za nasedačkou je asi lepší jet spíš u pravý strany. Potom až k dalšímu silničnímu mostu (400m) jedeme v pohodě na oči, pak prohlížíme. Je tam docela zablokovaná peřej, nejlepší lajna je u pravýho břehu a potom na střed. Další větší peřej opět prohlížíme, jede se ze středu doleva, pod mostem se jede u levý strany, za mostem se přejíždí zpátky na pravej střed. Hned po 100m je další most, kde opět prohlížíme. Čekáme na Kubu dost dlouho, prohlížel delší úsek. Najíždíme do vracáku u pravý strany na soutoku s nějakym potokem, potom se jede přes práh u vodočtu úplně u pravý strany, pokračuje se vpravo, dokud se nemine zprava šutr, za kterym je slušnej válec. Potom se jede na střed. Dál už je to trochu přehlednější a jedeme na oči. Je tam eště několik míst se solidníma válcema, ale dá se to objet / projet. Vysedáme ve vesnici Morgex na levym břehu u raftovýho centra. S prohlížením jedeme hodinu a čtvrt, dalo by se to dát za 30min.

\cara
Jedeme se podívat na Grand Eyviu, kde potkáváme Slávoše s Pražákama. Takhle odpoledne už to má docela slušně nakoupeno, takže to necháváme na ráno. Jedeme ještě na jednodušší raftovej úsek Dory.


\info[Dora Baltea]{17.8.}{vyšší stav}
\usek[pod Inferno kaňonem $N45.703994^\circ~E7.154570^\circ$]{Chavonne $N45.702905^\circ~E7.217788^\circ$}

Jedeme raftovej úsek, má to docela dost vody, takže je to hezký na zablbnutí, ale nic těžkýho. Já pendluju, ale nakonec se to všem docela líbí, tak dáváme druhej run, na kterej se přidám.

\cara

Spíme u Grand Eyvii na docela hezkym místě u cesty do lomu nad vodopádem $N45.644917^\circ~E7.274043^\circ$.


\info[Grand Eyvia]{18.8.}{střední stav, asi $33cm$ na vodočtu Cretaz}
\usek[Cretaz $N45.614506^\circ, E7.341663^\circ$]{Jez $N45.644914^\circ~E7.287377^\circ$}

Na nasedačce to vypadá, že to nemá moc vody, ale je to docela pohodovej střední stav. Po asi 500m se dojede do malýho kaňonu, kde jsou dvě výraznější peřeje. První je rozbitej slajdík, kterej se jede středem a pak stupeň s velkym bublákem vpravo, jede se vlevo. Pak je druhej stupeň ten se jede naopak ze středu doprava. Na konci kaňonu je vpravo jeskyně, která tvoří pekelnej sifon. Za týhle vody tam skoro nic neteče, ale za větší vody by si to chtělo dát pozor. Potom je nudnější rozlitej úsek, kterej se asi po 1km začne postupně zužovat a nabírá to spád. Je to docela hezkej úsek, nic zákeřnýho, všechno se dá jet na oči. Nejtěžší část je u konce. Za silničním mostem řeka nabere spád a je těžší, nejtěžších je posledních 300m mezi silniční galerií a vysedačkou. Je tu jeden cca 3m skok, kterej je dobrý prohlídnout, ale je v pohodě. Potom je takovej nepřehlednej úsek, kterej jedem u levý strany, jsou tu asi 2 docela solidní válce. Na konci těsně před vysedačkou jsou dva stupně s válcema. Oba se daj projet docela v pohodě, ale kousek pod nima je jez, kde de voda pod vratama, takže to chce jet opatrně. Za větší vody by tam mohly bejt poměrně solidní válce. 

Za týhle střední vody de přes hranu jezu jenom naprostý minimum vody, ale vrata na jezu sou pohyblivý, takže to neni úplně spolehlivej údaj.

Líbí se nám to, tak jedeme eště druhej run. Potom ještě jednou dáváme raftovej úsek na Doře, ale voda je výrazně nižší než předchozí den a je to docela nuda.

\cara

Odpoledne dáváme přejezd směrem na Inn. Jedeme po trase Aosta - Biella - Corte - Como - St. Moritz. Cesta je to neskutečně pekelná, v Itálii je to celý jedna velká vesnice, cestou jedem asi přes milión kruháčů. Pokud bych tu měl někdy jet znovu, tak jedině po dálnici, objíždět se to rozhodně nevyplatí. Na Inn dorážíme asi v půl třetí ráno.


\info[Inn]{19.8.}{$25m^3/s$ Tarasp}
\usek{Giarsun + Ardez}

Na závěr dáváme Inny, neni žádná velká voda, ale je to zábava :) Za cestu předjíždíme 5 jinejch part, je to tu jak na Vltavě. Pak už to jenom pálíme domu, cca v 10 dojíždíme do Budějc.

\newpage

\akce{Čerťáky}
\datum[22.8. - 23.8.]

Dorážim až v pátek večer. V sobotu jedeme 4 jízdy, v neděli jen 2, protože vyrážíme do Budějc přebalit se do našeho auta a vyrážíme na Isel.

Docela vydařená akce, vypustili letos 21 kubíků a je to docela znát, pěkná voda. Žádný drama se letos nekoná, největší peklo je vlak. Te sme si sice zaplatili (200Kč), takže jezdí častějc, ale nemá to valnej význam, protože jízdní řád se nestíhá a už tak 3. vlak jede úplně mimo stanovenej čas. Navíc jsou tam jenom tradiční 2 vagóny, takže je to peklo se tam narvat a stejně se tam občas někdo nevejde. N ato jak slibovali zajištěnou dopravu, tak je to bída.

\akce{Tyrolsko}
\datum[24.8. - 30.8.]

Kuba, Šuhy, Fíkuska

\info[Isel]{24.8.}{$260cm$ Lienz}
\usek{Pod kataraktem}

Jedeme $3\times$ s pendlem. Je docela málo vody, ale dá se to na pohodu.


\info[Defereggenbach]{24.8.}{$47cm$ Hopfgarten}
\usek{Wasserfallstrecke}

Voda je pro nás poměrně optimální, o moc víc bych mít nechtěl. Sice je to ze začátku trochu jebačka, ale dá se to sjet poměrně v pohodě. První drop má docela zajímavej nájezd, nejdřív takovej šikmej stupeň s válcem, kterej nás oba vypliv úplně u pravý strany. Potom je asi metrovej drop s válcem, ale oba sme ho projeli v pohodě, i když boofy nebyly ideální. Pak už následuje jenom skluzavka s kickerem a na konci parádní boof.

Druhej drop najíždíme úplně zprava po šikmym stupni. Kuba jede na drop úplně zprava na střed a dává parádní boof. Já si to najíždím víc na střed a jedu to víc směrem doprava, na kickeru trochu zadrhnu o dno a místo boofu padám po špičce, ale na pohodu. Potom už dojíždíme jenom krátkej kaňonek, kde je jeden pohodovej dropík a jeden dropík s undercutem vlevo a potom už jenom dojíždíme kousek na vysedačku u mostu, kde vylejzáme vlevo na silnici.


\info[Isel]{25.8.}{$260cm-270cm$ Lienz}
\usek{Pod kataraktem}

Jedeme zase $3\times$ s pendlem. Trochu to stoupá, ke konci je to i docela zábavný.


\info[Isel]{25.8.}{$105cm$ Hinterbichl}
\usek{Horní + Hinterbichl}

Nasedáme na drop v Hinterbichlu u mlejna. Neni tam moc vody, ale dá se to v pohodě. Drop je asi 4-5m, nikdo netrefujeme boof moc dobře, protože na hraně neni moc vody, ale je pod tim hloubka a dáváme docela na pohodu. Horní Isel má normální střední vodu, pohodička.


\info[Tauernbach]{26.8.}{$165cm$ Prosegg}
\usek{Standardstrecke}

Jedeme 2 jízdy s pendlem, první je poměrně málo vody, asi 155cm, ale dá se to jet docela na pohodu - já pendluju. Na druhou jízdu to docela nastoupalo, odhadem tak na 165cm. To už je v pohodě, skoro nikde z toho nečouhaj šutry a všechno je to dost na pohodu, fakt parádní poježdění. Jedeme asi 50 min.


\info[Defereggenbach]{26.8.}{$47cm$ Hopfgarten}
\usek{Tunnelstrecke}

Odpoledne jedeme na tunel. V první větší peřeji je strom napříč celou řekou, kterej přenášíme. Potom na oči jedeme až k peřeji nad mostem, kterou prohlížíme. Jedeme přes vracák vlevo, potom dropík levej střed a poslední drop úplně vpravo kolem skály, všichni jedem na pohodu. Slajdík pod mostem je super, až ke galerii je to v pohodě. Potom jistíme a prohlížíme peřej u galerie, lajna je tam poměrně jasná, najíždíme od pravýho břehu na střed přes první drop, potom se jede střed a až do konce s vodou. Všichni jedeme docela na pohodu, jenom Šuhy zvládne zapomenout lajnu a moc neví, kudy se do toho pustit. Ale nakonec to taky dává docela v klidu. Potom prohlížíme eště konec galerie, ale tam už za týhle vody nic dramatickýho neni. Jedeme až ke druhýmu mostu, dojezd už je WW3, ale neni to ani velká šutrovačka, na pohodu. S prohlížením a jištěním jedeme 1:40hod.


\info[Rienz]{27.8.}{standardní stav}
\usek[Mühlbach]{Brixen}

Nasedáme pod přehradou, stav je standardní, možná o maličko víc, než minule. První kilák až k nejtěžšímu kaňonu jedeme na pohodu bez většího prohlížení asi 30min. Nejtěžší místo jedeme jenom já s Kubou, oba docela na pohodu, když se dobře najede ten boof, tak už to člověka nějak vypláchne. Pokračujeme v podobnym tempu, v jednom hodně kamenitym místě se Petra hodně nepěkne zašprajcne mezi dva šutry a máčkne jí to pod vodu, naštěstí zůstane hlavou nad vodou. Po chvilce sem u ní a daří se mi jí vytáhnout. Je v pohodě, asi si pořádně neuvědomila, jak hodně blbá situace to byla. Potom už pokračujeme v poklidu až k mostu pod hradem, kde vystupuje Kuba a vrací se pro auto. My pokračujeme dál, řeka se postupně zklidňuje z WW3 na 1-2. Do Brixenu dojíždíme za 3 hodiny, jsme tam o trochu dřív, než Kuba. U hradu je to prej s lodí docela výšlap, prý je to snad lepší dojet.


\info[Boite]{28.8.}{vyšší střední, $7-12m^3/s$}
\usek[Most pod přehradou $46.416403^\circ N~12.243891^\circ E$]{Na druhý přehradě $46.413055^\circ N~12.318039^\circ E$}

Spíme na Boite někde u vesnice San Vito $46.477352^\circ N~12.193614^\circ E$ (konec horního úseku).

 Ráno jedeme kaňon mezi přehradama. Trochu nám dá zabrat najít vysedačku, nakonec se rozhodneme dojet až na přehradu, neni moc dlouhá. Nevíme jakej máme stav, ale určitě víc než minimálních 5 kubíků, tak se do toho pouštíme (ale radši bez Petry). Celkem brzo zjišťujeme, že máme takovou vyšší střední vodu. První zajímavý místo je asi 2m šikmej drop s válcem, trochu v něm svíčkujeme, ale všichni to dáme docela v pohodě. Potom začíná bejt řeka trochu víc z kopce a kontinuální, neni to žádný drama, ale na zastavování tam moc prostoru neni. Hned pár set metrů pod dropem je strom přes celou řeku, kterej přenášíme zprava. Potom přijíždíme k nepřeheldnýmu zavalenýmu místu, Kuba se ho snaží prohlídnout zleva, ale moc tam nevidí (při tom mu málem odplave loď, což by byl strašnej průser). Nakonec se mi podaří vystoupit vpravo a kouknout na to, je to docela v pohodě, jedeme celý zprava. Následuje několik set metrů v parádnim úzkym kaňonu bez větších peřejí, potom je další zavalená peřej. Prohlížíme z leva. Na konci je strom, ale je dostatečně nad vodou, aby se dal podjet. Místo najíždíme zleva, potom je cca 1m drop, jedeme u pravýho břehu, hned za nim je potřeba se sbalit pod strom. Následující větší peřej je u silničního mostu, je to takovej úzkej kanál u levýho břehu. V nájezdu má poměrně masivní válec, hned za tim kouká z vody takový žebro, kolem kterýho to padá do dalšího válce. Je to jetelný, ale válce sou hodně velký a eskymovat by tam člověk určitě nechtěl, je to hodně úzký. Nakonec se neodhodláme a přenášíme. Následuje ještě cca 1km větších peřejí, ale prohlížíme už jen zběžně, je to docela přehledný. Poslední asi 1km už je jenom dojezd do přehrady, vysedáme vlevo 100m před hrází u nějakýho přítoku.

Za zmínku stojí, že jsme nasedali odhadem za 7-8 kubíků a dojížděli jsme možná za 12ti. Přijde mi, že to nebylo jenom z přítoků, ale že spíš přehrada připustila. Ale těžko hádat. Rozhodně je dobrý brát to v úvahu, celej úsek je v kaňonu a výstup by byl extrémně složitej až nereálnej. Pozor je potřeba dát na stromy, bylo jich tam opravdu hodně, i když většina šla objet. Celkově bych to zhodnotil jako WW4 - 4+. Ale řeka je to super, nádhernej kaňon, neni to nijak extra těžký, mít čas, určitě bych to dal po druhý. Jeli jsme to necelý 3 hodiny.

\cara

Potom se jedeme podívat na Mae. Dlouho hledáme nasedačku a už je pozdě, tak to vzdáváme. Nicméně kaňon je úplně brutální, rozhodně to chci někdy jet. Nasedačka je asi 500m za lanovym mostem, je tam malá nenápadná pěšinka dolu k vodě. Nejdřív se mine dům vlevo, pak je nějakej dům vpravo a kousek za nim ta cestička odbočuje. S lodí to bude tak na hodinu, než se od parkoviště dojde k vodě.

Přejíždíme na super spaní u horolezecký stěny na Ahrnbachu $46.833907^\circ N~11.940157^\circ E$


\info[Ahrnbach]{29.8.}{$97cm$ St. Georgen}
\usek{Schloss}

Ráno jedem kouknout na horní Ahrnbach, minimum má bejt 90cm, ale je to blbost. Je to úplně suchý. Reálný minimum bych tipoval na 110cm. Jedem teda na Schloss, nemá to moc vody, ale jedeme s pendlem $2\times$. Všichni docela na pohodu, pořád je to v klidu jetelný a docela zábava. Je to sice hodně šutrovatý, ale lajna je všude a přes šutry se nemusí. Vysedáme za mostem nad jezem, dál to nemá smysl.


\info[Reinbach]{29.8.}{nižší střední}
\usek{Vodopády}

Jedeme omrknout Reinbach, má to spíš trochu nižší vodu, ale rozhodnem se to dát. Kuba jede první, já jdu s házečkou pod třetí drop, zbytek fotí. Kuba netrefuje první drop a zastavuje nad druhym vlevo. Neni moc vody, tak je tam plno času a v klidu si přejíždí úplně do prava, aby si najel druhej drop. Pak už má dobrou lajnu, vyzvedává hodně vlevo u skály, ale na pohodu. S přehledem dává i třetí drop. Du na to já, na boof se za tohodle stavu necejtim, takže jedu spíš na jistotu. Nájezd docela v pohodě, rovnou du na druhej drop, dopad jak nic, ale hned cejtim mokro na nohách a je mi to jasný - přišel sem o šprajdu. Vyplivne mě to dobře u pravýho břehu, vyzvedám hned, ještě se zkoušim chytit do malýho vracáku, kde by se dalo vylzet. Ale mám půl lodi vody a nestihnu to a zpátky proti proudu už to nejde. Je mi to celkem jasný, du ven, nechám odplavat loď a když je volno, Kuba mi ukáže a jedu poslední drop po prdeli. Je to celkem v klidu, válec podplavu, ale odplavat se odtamtut nedá, je tam docela silnej vracák. Kuba mě po chvilce vytáhne na házečce. 

Bohužel sem při tom zahodil pádlo, to chytlo dole levou lajnu a odplavalo, ale jinak všechno v pohodě. Jdem pak po turistický stezce, ale ta de úplně mimo vodu až dolu k vodopádům. Pak jdu ještě sám kus podél vody pod dva slajdy a vidim něco podezřelýho, ale nejsem si vůbec jistej. Sem tam sám, tak se radši vrátim k autu pro házečku a abych tam nelez sám. Všímám si, že znatelně víc vody, než když sme jeli. Když se vrátíme, nejsme si jistý, jestli je to pádlo, ale chci tam jít zkusit dojít. V tu chvíli vidíme, jak ho najednou veme voda - sice už je jasný, že to bylo moje pádlo - ale zároveň plave do prdele. Možná během 10ti - 15ti minut nastoupala voda fakt hodně (15 - 20cm), vodopády sou najednou úplně masivní, je na to potřeba dát pozor. Pádlo je v hajzlu, byla to velká smůla, že sem pro něj nedošel hned, když sem tam byl sám nebo že sme tam nebyli o 10min dřív, v pohodě bych pro něj došel... Příště ho nesmim vůbec pouštět, byla to trochu blbost.


\info[Isel]{30.8.}{$260cm$ Lienz}
\usek{Pod kataraktem}

Jedeme asi $3\times$ s pendlem a vyrážíme domu.

\cara
Schytáme ještě super objížďku kolem Zell am See (neni značená a musíme se vrátit do Mittersilu a jet přes Kitzbühl - wtf) ale domu dorážíme docela rozumně. Najeli sme cca jenom 1700km a vyšlo nás to na závratných 2300 na osobu. Super akce.


\akce{Podzimní Ötz}
\datum[19.9. - 20.9.]


\info[Otz]{19.9.}{$190cm$ Tumpen}
\usek{Lesní}

Na rozjetí dáváme lesní, ale je to docela bída, vody fakt málo.


\info[Venter Ache]{20.9.}{$190cm$ Tumpen}
\usek{Spodní}

Pak jedem $2\times$ spodní Venter, neni moc vody, ale je to celkem v pohodě poklidný poježdění.


\info[Otz]{19.9.}{$188cm$ Tumpen}
\usek{Mezimostí}

Odpoledne se připojuju na mezimostí s Ivošem a Kubou Weislem. Vstupenka se dá jet středem, nemusíme úplně vpravo. Tradičně je dole strom. Ivoš nechytá vracák a propadá někudy středem, ale na štěstí v pohodě.Až k prvnímu mostu docela v pohodě, akorát těsně před mostem mě to rozhodí a musim jet úplně doprava. Před druhym mostem trochu svícuju, ale v pohodě. Ke konci už sem celkem unavenej, protože jedeme skoro bez zastávky, zbytečně $2\times$ eskymuju, ale na pohodu. Vystupuju u mostu, kluci pokračujou Köfelsem.


\info[Venter Ache]{20.9.}{$190cm$ Tumpen}
\usek{Spodní}

V neděli dáváme 2 rychlý jízdy a vyrážíme včas domu. První jízdu sme našli Toro Kuby Suchýho, který mu uplavalo den předem na Heiligu.

\newpage

\akce{Podzimní Isel}
\datum[26.9. - 28.9.]

Prodlouženej víkend na Iselu s budějčákama, je nás soldiní parta - 10 lidí, 2 auta.


\info[Isel]{26.9.}{$80cm$ Hinterbichl}
\usek{Virgen}

Máme spodní limit, ale jetelný to docela je. První větší peřej je hned za prvním mostem, nejlepší lajna je středem přes šikmou desku, pak se jede k levýmu břehu a pokračuje se dál s vodou. Za další zatáčkou je druhá větší peřej, za tohodle stavu je to místy trochu jebačka, jedu celý středem, ale lepší lajna je asi u levý strany. Potom je několik set metrů klidnější úsek na oči, je tu jeden asi metrovej drop, kterej se najíždí zprava na střed a potom boof přes šutr. Další těžší místo je levá zatáčka, která končí v malym kaňonu. V nájezdu je šikmá šupna s několikametrovym převýšenim, potom 2 menší dropíky, který se jedou u pravý strany okolo napadanejch kmenů, pak to mizí v zatáčce do kaňonu.Za zatáčku neni moc vidět a prohlížení / přenášení je hodně složitý. Naštěstí je kaňon krátkej a dá se tam trochu nakouknout z vracáku vpravo. Jedinej potenciální problém jsou stromy, jinak je dole velkej vracák na odlovení. Na začátku se jede podél sklaního žebra středem přes válec, potom se to celý jede pravej střed přes dva válce. Za větší vody by tu mohly bejt hodně přísný válce.

Potom následuje jednodušší úsek, cca 1km, kterej se dá jet v klidu na oči. Ke konci úseku je most, pak se ještě kus jede, v ostrý pravý zatáčce je pak jedna větší peřej, která se jede, ale pod ní je potřeba zastavit. Následující peřej má hodně velkej spád a je hodně zablokovaná. Za týhle vody je to nejetelný, za větší by to možná šlo. Přenášíme vlevo a jedem jenom spodní část peřeje. Já v nájezdu beru šutr, rozhodí mě to, pak mě to bokem namáčkne na větev a nakonec eskymuju, takže pekelná lajna, ale naštěstí v pohodě. Potom je eště docela hezkej úsek jetelnej na oči, kterej postupně zvolňuje. Vysedáme na parkovišti před Matrei.

Rozhodně bych tu nechtěl mít míň vody, je to jetelný, ale docela jebačka. Dobrej pohodovej stav by moh bejt tak 85 - 90cm. S prohlíženim a přenášením jedeme necelý 3 hodiny.
\cara
Přejíždíme na Tauernbach, kde si procházíme Prosseggklamm a spíme u Tauernbachu kousek na nasedačkou.


\info[Tauernbach]{27.9.}{$146cm$ Prossegg}
\usek{Prosseggklamm}

Ráno vyrážíme jenom s Ondrou a Šafim. Nasedačka je pod první pekelnou peřejí, schází se tam korytem potoka. V prvním kaňonu jsou dva dropy, první jenom takovej malej šikmej, jede se středem. Druhej je vyšší (2m+) a neni tak čistej. Neni úplně kolmej, je potřeba ho najet zleva do prava na levej boof a pravej náklon. Pod nim to žene na skály vlevo. Já jedu s trochu blbym náklonem a zvrhá mě to doleva na skály, ale po chvilce vyzvedám těsně nad dalšim válečkem. Ondra najíždí v pohodě, Šafi se ale taky zvrhá, dostane pár šutrů do hlavy a nakonec plave. Naštěstí je v pohodě a v klidu vylejzá vpravo na konci kaňonu. Loď plave, tak jedeme za ní. Následující drop hned za kaňonem je asi nejstrašidelnější, je potřeba ho jet hodně vlevo. Nedá se tu ale nic moc vymejšlet, všechno odtejká a je docela na pohodu. Potom je tam ještě jeden asi 2m drop, jede se vlevo po šikmý šupně, ale je potřeba jet hodně směremdo prava, vlevo dole sou šutry. Tady i chytáme loď, ale má díru na špičce a Šafi tu končí. První kaňon doporučuju určitě prohlídnout z mostu a k posledním dvoum dropům se dá dojít až k vodě.

My s Ondrou pokračujeme dál. Po pár set metrech je první větší peřej (je vidět ze shora z cesty). Najíždí se boofem ze středu do prava, potom se jede pravej střed přes dva válce a končí se prudkou pravou zatáčkou s válcem, ale neni to žádný drama. Po pár set metrech je další peřej, která neni vidět z cesty. Jsou to takový žebra v řece, jede se to celý pravej střed a nakonci na středu boofem přes šutr. Ze břehu to vypadá dost šutrovatě, ale je to na pohodu, oba dáváme v pohodě. Hned za tim následuje hodně zablokovanej drop, jedeme vpravo mezi 2ma kickerama, lajna vypadá trochu strašidelně, ale nakonec je to úplně hladkej průjezd, pro mě nejzábavnější peřej. Poslední peřej je opět vidět z cesty ze shora, jede se celá u levý strany, potom asi 1m boof přes placák úplně vlevo. Za nim jedeme úplně doprava ke skále a potom přes malej šikmej šutr. Dál už jedeme na oči až k vysedačce, neni tu nic záludnýho. Vysedá se před ohromnym šutrem na pravym břehu, hned za tim následuje strašná smrt, vysedačku je určitě nutný prohlídnout předem! Odtud je potřeba vylezt nahoru na cestu a pěšky zpátky (15min) na nasedačku. S prohlíženim jedeme 40min.

Je to super úsek, měli sme spíš spodní limit, optimum by bylo tak 155cm, za větší vody by to asi dost sypalo a nesměl by tam nikdo zaplavat. Za projití stojí určitě i poslední kilometr kaňonu, kde to padá o několik set vejškovejch metrů v nádherný krajině.


\info[Defereggenbach]{27.9.}{$45cm$ Hopfgarten}
\usek{Wasserfallstrecke}

Odpoledne dáváme Defe. Začátek je klasická jebačka, ale vodopády super. Jede nás 9, takže neni nouze o dobrý lajny ani o carnage. Dobrej byl Ondra, na prvnim dropu mu zmizela půlka lodi v undercutu vlevo, ale vymotal se z toho v pohodě. Pete předvádí lajnu úplně vlevo po skále (neúmyslně), takže se úplně vyhne válci. Kupodivu je to celkem v pohodě. Jede týpek zvedá ve válci v nájezdu a pak zůstává ve válci pod dropem a plave. Já najíždím vstup i válec úplně v pohodě, ale končim víc vlevo než bych chtěl, tak si dávám špičku doprava na overboof. Dávám ale čelo a padnu tam dost bokem a zůstávám ve válci pod dropem. Zvedá se tu dost blbě, je to hodně propěněný. Rozhodně mě to jen tak nepustí, ale naštěsí nebojuju ani tak dlouho a pustí mě to, pod tim v klidu vyzvedám. Už sem i přemejšlel, že pudu ven, v tý propěněný vodě ve válci je člověk dost bezradnej.

Druhej drop najíždí víceméně všichni v pohodě, někdo jedena přímo, někdo přes pravej vracák, obě varianty v pohodě. Já dávám super boof úplně na plocho, docela mě překvapuje, že dopad v zádech celkem cejtim. Největší vzrušení je Pete, kterej z prdele zahazuje pádlo, ale rukama nezvedne.

Jedeme až dolu do vesnice, poslední kaňonek mě překvapuje, je tu eště jeden 1.5m super šikmej drop, potom jedna šikmá plotna s válcem. Poslední místo, kde bejval sifon, je úplně čistý, jede se úzkym průzkokem vlevo. Pak už tam nic neni.


\info[Saalach]{28.9.}{$22cm$ Unterjettenberg}
\usek{Lofer}

Cestou domu jedeme Lofer, je málo vody, ale je to pořád zábavný. Trojkombo je na pohodu, zbytek dáváme vtipný lajny různě mezi šutrama. Všichni jedeme na pohodu. Smutný je, že den předem se tu utopil nějakej Čech.


\end{document}.